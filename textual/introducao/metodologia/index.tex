\section[Metodologia]{Metodologia}

A primeira etapa do trabalho foi a idealização do modelo, onde foi levantada uma série de perguntas que culminaram em um profundo estudo para comprovar se era possível resolver o problema e de que forma a solução seria implementada. Adicionalmente, foram identificadas as tecnologias que poderiam ser usadas para ajudar no desenvolvimento da ferramenta proposta pelo modelo e facilitar sua replicação em diversas plataformas.

Em seguida, foi pensado na criação de uma interface para a ferramenta que fosse simples de usar, rodasse em serviços GraphQL já existentes e pudesse ser integrado em APIs REST que já possuem alguma formato de descrição de metadados.

Após isso, foi realizado um PoC\footnote{
  Proof of concept ou Prova de conceito
} para validar o protótipo do modelo. Em paralelo, iniciou-se o processo de escrita da monografia, implementação da ferramenta e preparação de um ambiente de validação que pudesse enfatizar bem os problemas reais que clientes Web tem sofrido devido a este acoplamento na busca de dados. 

Por fim, com o objetivo de apresentar uma comprovação da tese, foram postos para rodar os testes de validação, coletando os dados e apresentando os resultados analisados ao lado de ilustrações com gráficos.
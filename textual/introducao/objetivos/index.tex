\section[Objetivos]{Objetivos}

Com o objetivo de resolver o problema de acoplamento causado pela implementação do código de busca de clientes, este trabalho propõe um novo modelo de comunicação cliente-servidor que, através da automação de consultas em APIs Web, direcione desenvolvedores à implementação de um código de busca em clientes independente de especificação de API. Com isso, o modelo visa ser tolerante à mudanças na especificação, otimizar requisições e facilitar a composição de serviços em busca de dados. Para sua validação, será desenvolvido uma ferramenta baseado no modelo de comunicação proposto. \\

\textbf{Objetivos específicos} \\

No intuito de atingir o objetivo geral do trabalho, serão buscados os seguintes objetivos específicos:

\begin{itemize}
\item Propor modelo de comunicação replicável e agnóstico à plataforma.
\item Desenvolver ferramenta que aplique o modelo para plataforma Web.
\item Validar a ferramenta na comunicação entre clientes e APIs Web.
\item Promover a busca de dados através de linguagens de consulta.
\item Promover a documentação de APIs através de formatos de descrição. \\
\end{itemize}

\textbf{Limites da pesquisa} \\

O escopo do presente trabalho é delimitado da seguinte forma:

\begin{itemize}
\item Enfase na comunicação entre clientes web e APIs REST.
\item Enfase na tecnologia GraphQL para linguagem de consulta.
\item Enfase na tecnologia JSON Hyper-Schema para descrição de APIs.
\item Testes de composição de serviços não aplicados para a ferramenta.
\end{itemize}

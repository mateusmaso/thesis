\section[Objetivos]{Objetivos}

O principal objetivo deste trabalho é propor um modelo de comunicação cliente-servidor escalável que permita a composição de serviços e que faça a prevenção na criação de contrato entre clientes e interfaces de acesso. Para sua validação, será desenvolvido uma ferramenta que implemente o modelo e que seja tolerante à alterações no fluxo de dados de APIs. \\

\textbf{Objetivos específicos} \\

No intuito de atingir o objetivo geral do trabalho, serão buscados os seguintes objetivos específicos:

\begin{itemize}
\item Propor modelo de comunicação replicável e agnóstico a plataforma.
\item Desenvolver ferramenta que aplique o modelo para plataforma Web.
\item Validar a ferramenta na comunicação entre clientes e APIs Web.
\item Oferecer solução para consulta através da composição de serviços.
\item Promover a busca de dados através de linguagem de consulta.
\item Promover documentação e descrição de metadados em APIs. \\
\end{itemize}

\textbf{Limites da pesquisa} \\

O escopo do presente trabalho é delimitado da seguinte forma:

\begin{itemize}
\item Enfase na comunicação entre clientes e APIs REST.
\item Foco exclusivo nas tecnologias GraphQL, JSON e JSON Schema.
\item Testes de composição de serviços não aplicados para a ferramenta.
\end{itemize}

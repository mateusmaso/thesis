\section[Objetivos]{Objetivos}

Com o objetivo de resolver o problema de acoplamento mencionado anteriormente, este trabalho propõe um novo modelo de comunicação cliente-servidor através da automação na execução de consultas de dados sobre APIs Web. Dessa forma, o modelo visa direcionar desenvolvedores de clientes à implementação de códigos de busca independente de especificação de API. Resultando no desenvolvimento de clientes tolerante à mudanças na especificação.

Além disso, o modelo proposto prevê o desenvolvimento de uma ferramenta para servir como intermediador na comunicação e realizar a automação de consultas. Permitindo também a implementação de otimizações de requisição e a composição de serviços para busca de dados. \\

\textbf{Objetivos específicos} \\

No intuito de atingir o objetivo geral do trabalho, serão buscados os seguintes objetivos específicos:

\begin{itemize}
\item Propor modelo de comunicação replicável e agnóstico à plataforma.
\item Desenvolver ferramenta prevista pelo modelo para plataforma Web.
\item Validar a ferramenta na comunicação entre clientes e APIs Web.
\item Promover a busca de dados através de linguagens de consulta.
\item Promover a documentação de APIs através de formatos de descrição. \\
\end{itemize}

\textbf{Limites da pesquisa} \\

O escopo do presente trabalho é delimitado da seguinte forma:

\begin{itemize}
\item Enfase na comunicação entre clientes e APIs REST.
\item Enfase na tecnologia GraphQL para linguagem de consulta.
\item Enfase na tecnologia JSON Hyper-Schema para descrição de APIs.
\item Testes de composição de serviços não aplicados para a ferramenta.
\end{itemize}

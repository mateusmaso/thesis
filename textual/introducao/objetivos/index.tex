\section[Objetivos]{Objetivos}

Com o objetivo de resolver o problema de acoplamento mencionado anteriormente, este trabalho propõe um novo modelo de comunicação cliente-servidor através da automação na execução de consultas de dados sobre APIs Web.

O modelo visa direcionar desenvolvedores de clientes à implementação de códigos de busca independente de especificação de API, resultando no desenvolvimento de clientes tolerantes às mudanças na especificação. Ainda, prevê o desenvolvimento de uma ferramenta que funciona como intermediadora na comunicação e automação de consultas, permitindo otimização de requisições e composição de serviços para consulta de dados. \\

\textbf{Objetivos específicos} \\

No intuito de atingir o objetivo geral do trabalho, pretende-se atender os seguintes objetivos específicos:

\begin{itemize}
\item Propor modelo de comunicação replicável e agnóstico à plataforma.
\item Desenvolver ferramenta prevista pelo modelo para plataforma Web.
\item Validar a ferramenta na comunicação entre clientes e APIs Web.
\item Promover a consulta de dados através de linguagens de consulta.
\item Promover a documentação de APIs através de formatos de descrição. \\
\end{itemize}

\textbf{Limites da pesquisa} \\

Dentro do escopo do trabalho, está delimitado os itens:

\begin{itemize}
\item Enfase na comunicação entre clientes e APIs REST.
\item Enfase na tecnologia GraphQL para linguagem de consulta.
\item Enfase na tecnologia JSON Hyper-Schema para descrição de APIs.
\item Testes de composição de serviços não aplicados para a ferramenta.
\end{itemize}

\section[Descrição do Problema]{Descrição do Problema}

O mercado de APIs Web transformou o modo de comunicação entre aplicações distribuídas para busca de informações e execução de operações na World Wide Web. Não é de hoje que organizações têm se preocupado em disponibilizar APIs de serviços para consumo próprio ou de terceiros. Empresas como Facebook e Netflix mostram que construir uma aplicação em forma de serviço e disponibilizar uma interface de acesso é essencial para entrar rápido no mercado de plataformas emergentes, oferecer uma melhor experiência para o usuário através do desenvolvimento de clientes nativos, além de agregar valor em seu modelo de negócio ao disponibilizar dados e operações para o uso de terceiros. \cite{Art2016}

Segundo Duvander, um dos motivos que contribuiu para o aumento do número de APIs de serviços foi após a introdução em 2001 do estilo de arquitetura REST. Contudo, após 15 anos de sua introdução e diversos clientes escritos com base em seu modelo de comunicação, REST tem-se mostrado uma solução ineficiente para lidar com o acesso de um grande número de clientes diversificados. Isso porque sua API, após publicada, não prevê mudança no contrato em busca de suprir a constante transformação na demanda de dados por clientes. \cite{Duvander2013-2}

Por outro lado, novos estilos de arquitetura surgem com o objetivo de minimizar parte deste problema de contrato, mas encontram difícil adoção por clientes já existentes que fazem a comunicação de APIs REST. Uma vez que estes, precisam quebrar o contrato atual e passar por uma custosa transição e reimplementação no seu código de busca.

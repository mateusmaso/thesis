\section[Descrição do Problema]{Descrição do Problema}

Em 2005, ocorreu uma grande transição no modelo de comunicação entre aplicações distribuídas, onde estas passaram a utilizar amplamente o protocolo HTTP e o modelo cliente-servidor para a troca de informações na World Wide Web. Um dos principais motivos que contribuiu na época para esta crescente transição foi devido a facilidade de aplicações em expor sua API através do estilo de arquitetura REST, assim como clientes em se comunicar com essas interfaces remotas. \cite{Duvander2013-2}

Hoje, empresas como Facebook e Netflix mostram que, independente do estilo de arquitetura, construir APIs Web é, não apenas essencial para entrar rápido no mercado de plataformas emergentes, como também um novo meio de agregar valor em seu próprio modelo de negócio e oferecer uma melhor experiência a seus usuários. \cite{Art2016}

Contudo, após anos de sua popularização, serviços têm mostrado dificuldades em realizar mudanças em suas APIs Web sem comprometer a comunicação de clientes. Isso porque clientes continuam implementando seu código de busca através de chamadas diretamente na API, ocasionando um acoplamento da especificação muitas vezes indesejado pelos serviços. Além de promover versionamento de APIs, este acoplamento tem atrasado a adoção de novas tecnologias e estilos de arquitetura por serviços.

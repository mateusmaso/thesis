\section[Descrição do Problema]{Descrição do Problema}

Em 2005, ocorreu uma grande transição no modelo de comunicação entre aplicações distribuídas, onde estas passaram a utilizar amplamente o protocolo HTTP e o modelo cliente-servidor para a troca de informações na World Wide Web. Um dos principais motivos que contribuiu na época para esta crescente transição, segundo Duvander, foi devido a facilidade de aplicações em expor sua API através do estilo de arquitetura REST, assim como clientes em se comunicar com essas APIs. \cite{Duvander2013-2}

Hoje, empresas como Facebook e Netflix mostram que, independente do estilo de arquitetura, construir APIs Web é, não apenas essencial para entrar rápido no mercado de plataformas emergentes, como também um novo meio de agregar valor em seu próprio modelo de negócio e oferecer uma melhor experiência a seus usuários. \cite{Art2016}

Contudo, após anos de sua popularização e diversas implementações em clientes com base em seu modelo de comunicação, serviços têm mostrado dificuldades em realizar mudanças em suas APIs sem comprometer a comunicação. Isso porque clientes estão implementando em seu código de busca chamadas de forma direta à especificação de APIs Web, ocasionando um acoplamento muitas vezes indesejado pelos serviços.

Além disso, este acoplamento tem dificultado a adoção de novas tecnologias e estilos de arquitetura por serviços. que necessitam de alterações na especificação da API, pois não conseguem convencer seus benefícios em troca da custosa transição e reimplementação no seu código de busca que clientes vão precisar passar.

\chapter{GraphQL}

GraphQL é, além de um interpretador, uma linguagem de consulta de dados criada pelo Facebook para trabalhar com APIs de forma alternativa. Seu objetivo é fornecer uma descrição completa e compreensível de dados disponíveis em interfaces de aplicação, permitindo que clientes façam consultas de forma precisa em busca de dados que desejam trabalhar. \cite{GraphQL2016}

Por ser uma especificação recente, após sua publicação em 2015, GraphQL já apresenta implementações em diversas linguagens de programação e atualmente é utilizado em diversos contextos, como na comunicação entre cliente-servidor, microsserviços, navegação de árvores, gerador de consultas para banco de dados, entre outros.

É importante ressaltar que GraphQL não é uma linguagem de banco de dados, por mais que possa ser utilizada para esta finalidade. Ao invés, sua linguagem e interpretador trabalham com a ideia de mapeamento de campos e tipos de dados de retorno em APIs, fornecendo um esquema para interagir com interfaces através da execução de consultas da linguagem. \cite{GraphQL2016}

Nas figuras 8 e 9, é descrito um exemplo de mapeamento de uma API em JavaScript em um esquema GraphQL, através da análise de estruturas de retorno.

\begin{figure}[H]
  \centering
  \begin{minted}[frame=single,framesep=10pt,fontsize=\footnotesize]{text}
    var pessoa = {
      nome: "Mateus Maso"
    }

    class Query {
      pessoa() {
        return pessoa
      }
    }

    class Pessoa {
      id(pessoa) {
        return pessoa.id
      }
      
      nome(pessoa) {
        return pessoa.nome
      }
    }
  \end{minted}
  \caption{Exemplo de API em JavaScript (ES6)}
\end{figure}

\begin{figure}[H]
  \centering
  \begin{minted}[frame=single,framesep=10pt,fontsize=\footnotesize]{text}
    type Query {
      pessoa: User
    }

    type Pessoa {
      id: ID
      nome: String
    }
  \end{minted}
  \caption{Esquema GraphQL para API da Figura 8}
\end{figure}

Após criado o esquema, é descrito nas figuras 10 e 11 um exemplo de consulta utilizando a sintaxe GraphQL e os tipos de dados mapeados pelo esquema. Ao ser executada, a ferramenta irá analisar, validar e transformar a consulta em chamadas para a API, retornado uma representação exata dos dados requisitados no formato JSON. \cite{GraphQL2016}

\begin{figure}[H]
  \centering
  \begin{minted}[frame=single,framesep=10pt,fontsize=\footnotesize]{text}
    query {
      pessoa {
        nome
      }
    }
  \end{minted}
  \caption{Exemplo de consulta GraphQL para Figura 9}
\end{figure}

\begin{figure}[H]
  \centering
  \begin{minted}[frame=single,framesep=10pt,fontsize=\footnotesize]{text}
    {
      "pessoa": {
        "nome": "Mateus Maso"
      }
    }
  \end{minted}
  \caption{Resposta JSON esperada após execução pela Figura 10}
\end{figure}

A seguir, será feita uma abordagem sobre os elementos que compõem a linguagem, seu sistema de tipagem e como analisar metadados de esquemas utilizando o processo de introspecção.

\section[Linguagem de Consulta]{Linguagem de Consulta}

Clientes que buscam realizar consultas de dados em esquemas GraphQL, antes precisam entender seu documento de requisição. Um documento GraphQL contém operações de consultas ou mutações, além de unidades de composição e reuso de consultas, descritas pelo nome de fragmentos. \cite{Facebook2016} \\

\textbf{Sintaxe} \\

Documentos GraphQL são inspirados no formato de estrutura de dados JSON, porém sem seus valores e com alterações na sintaxe. Para que um esquema entenda um documento GraphQL, este deve descrever ao menos uma operação de consulta ou mutação. Uma operação de consulta é um processo de leitura da API. Já uma operação de mutação é representada por dois processos, uma escrita seguida de uma leitura na API.

Ao executar uma operação, deve-se expressar um conjunto de dados (seleção) que se deseja receber. Este conjunto é representado por campos e fragmentos, visto na figura 11, onde um campo pode receber argumentos e deve descrever um dado ou subconjunto de dados. Isso permite explorar relacionamentos complexos através do profundo aninhamento de conjuntos de seleções em busca de retornar uma estrutura JSON parecida com o que se escreve na linguagem.

\begin{figure}[H]
  \centering
  \begin{minted}[frame=single,framesep=10pt,fontsize=\footnotesize]{text}
    query {
      pessoa(id: 4) {
        id
        nome
        sobrenome
        nascimento: aniversario {
          mes
          dia
        }
        amigos(limite: 10) {
          nome
        }
      }
    }
  \end{minted}
  \caption{Seleção de campos em consultas GraphQL}
\end{figure}

\textbf{Fragmentos} \\

Fragmentos são a principal unidade de composição em GraphQL, pois permitem o reuso de seleção de campos que se repetem em documentos. Um fragmento é representado por um nome, seguido pelo tipo que está sendo aplicado à seleção de campos. Podem também ser expressos de forma "inline", onde não há a necessidade de definir um nome.

\begin{figure}[H]
  \centering
  \begin{minted}[frame=single,framesep=10pt,fontsize=\footnotesize]{text}
    query {
      pessoa(id: 4) {
        id
        ...identidade
        ...relacionamentos
      }
    }

    fragment identidade on Pessoa {
      nome
      sobrenome
      nascimento: aniversario {
        mes
        dia
      }
    }

    fragment relacionamentos on Pessoa {
      amigos(limite: 10) {
        nome
      }
    }
  \end{minted}
  \caption{Consulta da Figura 12 utilizando fragmentos}
\end{figure}

\section[Sistema de Tipagem]{Sistema de Tipagem}

Para descrever um esquema GraphQL a partir de uma API, antes é preciso fazer o uso de seu sistema de tipagem. Através da abstração de entidades de um serviço, representa-se um conjunto finito de tipos para ser acessado por documentos GraphQL. Um tipo é uma forma de representação de dados específico da linguagem de definição, que indica à ferramenta como interpretar operações de consulta e mutação.

Durante o processo de mapeamento de entidades, faz-se a conversão de estruturas de dados em conjuntos dos tipos folha, de acondicionamento, de composição e abstratos. Um tipo folha é representado por valores finais, singulares e não nulos da estrutura, onde os tipos de acondicionamento e composição são utilizados em cima dos tipos folha para alterar o comportamento e combinar outros tipos em busca da representação de estruturas mais complexas. Por fim, existem os tipos abstratos, que servem para reusar os tipos anteriores através do uso de conceitos de alto nível.

\begin{table}[H]
  \centering
  \begin{tabular}{|c|c|c|c|c|}
    \hline
    Tipo & Classificação \\
    \hline
    Enum & Folha \\
    \hline
    Int & Folha \\
    \hline
    Float & Folha \\
    \hline
    String & Folha \\
    \hline
    Boolean & Folha \\
    \hline
    ID & Folha \\
    \hline
    Object & Composição \\
    \hline
    Union & Abstrato \\
    \hline
    Interface & Abstrato \\
    \hline
    Non-Null & Acondicionamento \\
    \hline
  	List & Acondicionamento \\
    \hline
  \end{tabular}
  \caption{Classificação de tipos GraphQL}
\end{table}

Após a conversão de estruturas de dados em tipos GraphQL, para que estes também sejam acessados por documentos, é preciso incluí-los em um tipo de composição especial definido pelo esquema chamado "root". Dessa forma, é possível modelar os primeiros campos de acesso que o esquema deseja disponibilizar para que clientes façam a execução de operações a partir deles.

\begin{figure}[H]
  \centering
  \begin{minted}[frame=single,framesep=10pt,fontsize=\footnotesize]{text}
    interface Individuo {
      nome: String
    }

    type Pessoa implements Individuo {
      foto: Foto
      amigos: [Pessoa]
    }

    type Foto {
      url: String
    }

    union Resultado = Foto | Pessoa

    type Pesquisa {
      resultado: Resultado
    }
  \end{minted}
  \caption{Exemplo de representação do esquema da Figura 17}
\end{figure}

\begin{figure}[H]
  \centering
  \begin{minted}[frame=single,framesep=10pt,fontsize=\footnotesize]{text}
    var IndividuoType = new GraphQLInterfaceType({
      fields: {
        nome: {type: GraphQLStringType}
      }
    })

    var PessoaType = new GraphQLObjectType({
      interfaces: [IndividuoType],
      fields: () => {
        return {
          foto: {type: FotoType},
          amigos: {type: [PessoaType]}
        }
      }
    })

    var FotoType = new GraphQLObjectType({
      fields: {
        url: {type: GraphQLStringType}
      }
    })

    var ResultadoType = new GraphQLUnionType({
      types: [PessoaType, FotoType]
    })

    var PesquisaType = new GraphQLObjectType({
      fields: {
        resultado: {type: ResultadoType}
      }
    })

    var schema = new GraphQLSchema({
      query: new GraphQLObjectType({
        fields: {
          pesquisa: {type: PesquisaType}
        }
      })
    })
  \end{minted}
  \caption{Implementação JavaScript da Figura 17}
\end{figure}

\section[Introspecção]{Introspecção}

Em GraphQL, o termo introspecção refere-se à uma consulta especial usada na busca por metadados de esquemas. Uma das vantagens desta funcionalidade está na possibilidade de ferramentas usá-la para gerar código, documentação, prever mudanças, alertar de campos deprecados, entre outras aplicações. Isso torna GraphQL uma ótima opção como estilo de arquiteturas em sistemas distribuídos, pois apresenta de forma "nativa" uma poderosa linguagem de descrição de serviços.

\begin{figure}[H]
  \centering
  \begin{minted}[frame=single,framesep=10pt,fontsize=\footnotesize]{javascript}
    query introspeccao {
      __schema {
        queryType { name }
        mutationType { name }
        types {
          kind
          name
          description
          fields {
            name
            description
          }
        }
      }
    }
  \end{minted}
  \caption{Introspecção}
\end{figure}


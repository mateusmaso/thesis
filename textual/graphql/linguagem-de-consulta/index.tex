\section[Linguagem de Consulta]{Linguagem de Consulta}

Clientes que buscam realizar consultas de dados em esquemas GraphQL, antes precisam entender seu documento de requisição. Um documento GraphQL contém operações de consultas ou mutações, além de unidades de composição e reuso de consultas, descritas pelo nome de fragmentos. \cite{Facebook2016} \\

\textbf{Sintaxe} \\

Documentos GraphQL são inspirados no formato de estrutura de dados JSON, porém sem seus valores e com alterações na sintaxe. Para que um esquema entenda um documento GraphQL, este deve descrever ao menos uma operação de consulta ou mutação. Uma operação de consulta é um processo de leitura da API. Já uma operação de mutação é representada por dois processos, uma escrita seguida de uma leitura na API.

Ao executar uma operação, deve-se expressar um conjunto de dados (seleção) que se deseja receber. Este conjunto é representado por campos e fragmentos, visto na figura 11, onde um campo pode receber argumentos e deve descrever um dado ou subconjunto de dados. Isso permite explorar relacionamentos complexos através do profundo aninhamento de conjuntos de seleções em busca de retornar uma estrutura JSON parecida com o que se escreve na linguagem.

\begin{figure}[H]
  \centering
  \begin{minted}[frame=single,framesep=10pt,fontsize=\footnotesize]{text}
    query {
      pessoa(id: 4) {
        id
        nome
        sobrenome
        nascimento: aniversario {
          mes
          dia
        }
        amigos(limite: 10) {
          nome
        }
      }
    }
  \end{minted}
  \caption{Seleção de campos em consultas GraphQL}
\end{figure}

\textbf{Fragmentos} \\

Fragmentos são a principal unidade de composição em GraphQL, pois permitem o reuso de seleção de campos que se repetem em documentos. Um fragmento é representado por um nome, seguido pelo tipo que está sendo aplicado à seleção de campos. Podem também ser expressos de forma "inline", onde não há a necessidade de definir um nome.

\begin{figure}[H]
  \centering
  \begin{minted}[frame=single,framesep=10pt,fontsize=\footnotesize]{text}
    query {
      pessoa(id: 4) {
        id
        ...identidade
        ...relacionamentos
      }
    }

    fragment identidade on Pessoa {
      nome
      sobrenome
      nascimento: aniversario {
        mes
        dia
      }
    }

    fragment relacionamentos on Pessoa {
      amigos(limite: 10) {
        nome
      }
    }
  \end{minted}
  \caption{Consulta da Figura 12 utilizando fragmentos}
\end{figure}

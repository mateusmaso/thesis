\section[Introspecção]{Introspecção}

Em GraphQL, o termo introspecção refere-se a uma consulta especial usada na busca por metadados de esquemas. Uma das vantagens desta funcionalidade está na possibilidade de ferramentas a usarem para gerar código, documentação, prever mudanças, alertar de campos obsoletos (\textit{deprecated}), entre outras aplicações. Isso torna GraphQL uma ótima opção como estilo de arquiteturas em sistemas distribuídos, pois apresenta de forma "nativa" uma poderosa linguagem de descrição de serviços.

\begin{figure}[H]
  \centering
  \begin{minted}[frame=single,framesep=10pt,fontsize=\footnotesize]{javascript}
    query introspeccao {
      __schema {
        queryType { name }
        mutationType { name }
        types {
          kind
          name
          description
          fields {
            name
            description
          }
        }
      }
    }
  \end{minted}
  \caption{Introspecção}
\end{figure}

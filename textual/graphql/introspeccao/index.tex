\section[Introspecção]{Introspecção}

Em GraphQL, o termo introspecção refere-se a uma consulta especial usada na busca por metadados em esquemas. Uma das vantagens desta funcionalidade em sistemas distribuídos está no seu uso para descrever serviços que expõe esquemas GraphQL para acesso de clientes. Estes, por sua vez, podem usar a introspecção para inspecionar documentação, alertar sobre campos obsoletos (\textit{deprecated}), gerar código, entre outras possibilidades.

\begin{figure}[H]
  \centering
  \begin{minted}[frame=single,framesep=10pt,fontsize=\footnotesize]{javascript}
    query introspeccao {
      __schema {
        queryType { name }
        mutationType { name }
        types {
          kind
          name
          description
          fields {
            name
            description
          }
        }
      }
    }
  \end{minted}
  \caption{Consulta de introspecção}
\end{figure}

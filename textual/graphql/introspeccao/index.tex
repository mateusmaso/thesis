\section[Introspecção]{Introspecção}

Em GraphQL, o termo introspecção refere-se a uma consulta especial usada na busca por metadados de esquemas. Uma das vantagens desta funcionalidade está no seu uso em sistemas distribuídos, pois oferece uma alternativa "nativa" aos demais formatos para descrição de APIs REST.

% para descrição de acesso.

% documentação

% que disponibilizam seu esquema.

% para prever mudanças, alertar de campos obsoletos (\textit{deprecated}), gerar código, entre outras possibilidades. Isso torna GraphQL uma ótima opção para ser usado em sistemas distribuídos, 

\begin{figure}[H]
  \centering
  \begin{minted}[frame=single,framesep=10pt,fontsize=\footnotesize]{javascript}
    query introspeccao {
      __schema {
        queryType { name }
        mutationType { name }
        types {
          kind
          name
          description
          fields {
            name
            description
          }
        }
      }
    }
  \end{minted}
  \caption{Consulta de introspecção}
\end{figure}

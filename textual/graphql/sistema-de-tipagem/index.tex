\section[Sistema de Tipagem]{Sistema de Tipagem}

Para descrever um esquema GraphQL a partir de uma API, antes é preciso fazer o uso de seu sistema de tipagem. Através da abstração de entidades de um serviço, representa-se um conjunto finito de tipos para ser acessado por documentos GraphQL. Um tipo é uma forma de representação de dados especifico da linguagem de definição para que a ferramenta saiba como interpretar operações de consulta e mutação.

Durante o processo de mapeamento de entidades, faz-se a conversão de estruturas de dados em conjuntos dos tipos folha, de acondicionamento, de composição e abstratos. Um tipo folha é representado por valores finais, singulares e não nulos da estrutura, onde os tipos de acondicionamento e composição são utilizados em cima dos tipos folha para alterar o comportamento e combinar outros tipos em busca da representação de estruturas mais complexas. Por fim, existem os tipos abstratos, que servem para reusar os tipos anteriores através do uso de conceitos alto nível.

\begin{table}[H]
  \centering
  \begin{tabular}{|c|c|c|c|c|}
    \hline
    Tipo & Classificação \\
    \hline
    Enum & Folha \\
    \hline
    Int & Folha \\
    \hline
    Float & Folha \\
    \hline
    String & Folha \\
    \hline
    Boolean & Folha \\
    \hline
    ID & Folha \\
    \hline
    Object & Composição \\
    \hline
    Union & Abstrato \\
    \hline
    Interface & Abstrato \\
    \hline
    Non-Null & Acondicionamento \\
    \hline
  	List & Acondicionamento \\
    \hline
  \end{tabular}
  \caption{Classificação de tipos GraphQL}
\end{table}

Após a conversão de estruturas de dados em tipos GraphQL, para que estes também sejam acessados por documentos, é preciso incluí-los em um tipo de composição especial definido pelo esquema chamado "root". Dessa forma, é possível modelar os primeiros campos de acesso que o esquema deseja disponibilizar para que clientes façam a execução de operações a partir deles.

\begin{figure}[H]
  \centering
  \begin{minted}[frame=single,framesep=10pt,fontsize=\footnotesize]{text}
    interface Individuo {
      nome: String
    }

    type Pessoa implements Individuo {
      foto: Foto
      amigos: [Pessoa]
    }

    type Foto {
      url: String
    }

    union Resultado = Foto | Pessoa

    type Pesquisa {
      resultado: Resultado
    }
  \end{minted}
  \caption{Exemplo de representação do esquema da Figura 17}
\end{figure}

\begin{figure}[H]
  \centering
  \begin{minted}[frame=single,framesep=10pt,fontsize=\footnotesize]{text}
    var IndividuoType = new GraphQLInterfaceType({
      fields: {
        nome: {type: GraphQLStringType}
      }
    })

    var PessoaType = new GraphQLObjectType({
      interfaces: [IndividuoType],
      fields: () => {
        return {
          foto: {type: FotoType},
          amigos: {type: [PessoaType]}
        }
      }
    })

    var FotoType = new GraphQLObjectType({
      fields: {
        url: {type: GraphQLStringType}
      }
    })

    var ResultadoType = new GraphQLUnionType({
      types: [PessoaType, FotoType]
    })

    var PesquisaType = new GraphQLObjectType({
      fields: {
        resultado: {type: ResultadoType}
      }
    })

    var schema = new GraphQLSchema({
      query: new GraphQLObjectType({
        fields: {
          pesquisa: {type: PesquisaType}
        }
      })
    })
  \end{minted}
  \caption{Implementação JavaScript da Figura 17}
\end{figure}

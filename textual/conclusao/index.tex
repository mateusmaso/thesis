\chapter{Conclusão}

Com aumento no número de serviços e diversidade de clientes distribuídos pela Web, tornou-se um desafio de desenvolvimento conseguir satisfazer a demanda de dados de clientes em serviços sem estar preso à contratos de acesso. Ao mesmo tempo que novos estilos de arquitetura surgem para minimizar este problema, estes não apresentam compatibilidade aos estilos mais usados no mercado. Dessa forma, surgindo a necessidade de um novo modelo de comunicação cliente-servidor que pudesse simplificar o acesso e composição de dados em serviços independente de seu fluxo de acesso, além de não apresentar ser facilmente vulnerável à suas mudanças ao longo do tempo.

Este trabalho realizou um estudo sobre a causa e propôs um melhor modelo de comunicação entre as duas partes interessadas. Além de mostrar a importância de criar um ambiente de desenvolvimento que promove mudanças no fluxo de dados e estilo de arquitetura para facilitar a comunicação.

Desta forma, uma das principais contribuições deste trabalho foi em mostrar que deveriamos estar nos preocupando em não como acessar os dados/entender APIs, mas sim expressar as dependecias que precisamos em cada cliente através de uma linguagem como GraphQL. E deixar que ferramentas automatizem a realização da busca desses dados. O importante é não estarmos desenvolvendo clientes acoplados a APIs de serviços, pois isso dificulta X, Y e Z.

Além disso, este trabalho mostra um ótimo exemplo de outras oportunidades que podem ser criadas a partir da documentação de APIs em linguagens de acesso à maquinas. E que devemos escolher estilos de arquitetura que melhor se encaixam no contexto de desenvolvimento de um serviço. Por exemplo, GraphQL (interpretador) é ótimo para diversos casos, mas muito verboso para um projeto inicial. REST, é ótimo para debugar e está no mercado há muito mais tempo.

\section[Trabalhos Futuros]{Trabalhos Futuros}

\begin{itemize}
  \item Realização de testes de validação para a composição de serviços.
  \item Criação de novos adaptadores da ferramenta para formatos de descrição de APIs. (OpenAPI, RAML, API Blueprint)
  \item Implementação da especificação da ferramenta em outras plataformas de desenvolvimento. (Mobile, Desktop, etc)
  \item Aprimoramento do algoritmo de análise de consultas.
\end{itemize}

Acredita-se que após a analise dos resultados obtidos pela ferramenta, novas.

\begin{enumerate}
\item Realizar mais testes de validação na transição versões de APIs e estilos de arquitetura
\item Realizar mais testes de composição de dados em serviços/microserviços.
\item Criar novos adaptadores de metadados para formatos de descrição de APIs.
\item Replicar a ferramenta em outras plataformas de desenvolvimento de clientes como mobile.
\item Melhorar o algoritmo de busca e análise de dados, para diminuição do tempo de overhead.
\end{enumerate}

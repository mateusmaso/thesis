\chapter{Conclusão}

Tornou-se evidente, através deste trabalho, que o acoplamento entre o código de busca e especificação de APIs Web têm dificultado serviços em oferecer uma melhor comunicação. Implementação % evitar versionamento

%  busca de uma solução para o desacoplamento 

O esforço dedicado neste trabalho em explorar esta área resultou no desenvolvimento de um novo modelo de comunicação. Este buscou eliminar a preocupação de clientes em estar atualizando seu código de busca a cada mudança na API. Através de uma implementação oferecer uma implementação do código de busca não dependente à especificação de API, 

Uma das principais contribuição deste trabalho foi mostrar os benefícios da automação na execução de consultas de dados e potencial facilidade na composição de serviços que a ferramenta pode proporcionar, mesmo não sendo validada. Foi percebida a importância de se escrever um código de busca através de linguagens de consulta como GraphQL e de serviços em disponibilizar metadados de API.

Por fim, os benefícios do uso desse modelo será apenas sentido por serviços se estes encorajar o uso da ferramenta por seus clientes. Onde o tempo pode demorar baseado no numero de tipos de clientes ativos fazendo comunicação. % existentes

\section[Trabalhos Futuros]{Trabalhos Futuros}

\begin{itemize}
  \item Realização de testes de validação para a composição de serviços.
  \item Criação de novos adaptadores da ferramenta para formatos de descrição de APIs. (OpenAPI, RAML, API Blueprint)
  \item Implementação da especificação da ferramenta em outras plataformas de desenvolvimento. (Mobile, Desktop, etc)
  \item Aprimoramento do algoritmo de análise de consultas.
\end{itemize}

\begin{itemize}
\item Realizar mais testes de validação como a de composição de serviços, transição entre versões de API e estilos de arquitetura.
\item Criar novos adaptadores da ferramenta para formatos de descrição de APIs. (OpenAPI, RAML, API Blueprint)
\item Implementar a especificação da ferramenta em outras plataformas de desenvolvimento. (Android, iOS, etc)
\item Melhorar o algoritmo de análise de consultas AST.
\end{itemize}

\chapter{Conclusão}

Tornam-se evidentes, através deste trabalho, as dificuldades de serviços em realizar mudanças na especificação de APIs uma vez que estas já possuem clientes fazendo o seu acesso. Além disso, nota-se que é possível explorar novos modelos de comunicação a fim de manter a eficiência da comunicação cliente-servidor evitando versionamento.

O esforço dedicado neste trabalho em explorar esta área culminou no desenvolvimento de um novo modelo de comunicação. Buscou-se eliminar a preocupação de clientes em estar continuamente atualizando seu código de busca a cada mudança na API e direcionar desenvolvedores de clientes à implementação de códigos de busca independentes de especificação de API.

Apesar das vantagens do modelo, existe um investimento a mais com a integração da ferramenta para que clientes e serviços possam usufruir os benefícios demonstrados nos resultados dos testes de validação. Felizmente, serviços que disponibilizam descrições de metadados da sua API apresentam uma menor barreira de entrada e já podem ser consultados utilizando a ferramenta.

Por fim, uma das principais contribuições deste trabalho é estampar os benefícios da automação na execução de consultas de dados e composição de serviços que o modelo e a ferramenta podem proporcionar. Basta que os desenvolvedores de clientes se preocupem em escrever um código de busca através de linguagens de consulta como o GraphQL e dos serviços em disponibilizarem uma completa descrição dos metadados de APIs.

\section[Trabalhos Futuros]{Trabalhos Futuros}

\begin{itemize}
  \item Realização de testes de validação para a composição de serviços.
  \item Criação de novos adaptadores da ferramenta para formatos de descrição de APIs. (OpenAPI, RAML, API Blueprint)
  \item Implementação da especificação da ferramenta em outras plataformas de desenvolvimento. (Mobile, Desktop, etc)
  \item Aprimoramento do algoritmo de análise de consultas.
\end{itemize}

Identificamos as seguintes possibilidades de continuidade e evolução deste trabalho: 

\begin{itemize}
\item Realização de testes de validação para a composição de serviços.
\item Criação de novos adaptadores da ferramenta para formatos de descrição de APIs. (OpenAPI, RAML, API Blueprint)
\item Implementação da especificação da ferramenta em outras plataformas de desenvolvimento. (Mobile, Desktop, etc)
\item Aprimoramento do algoritmo de análise de consultas.
\end{itemize}

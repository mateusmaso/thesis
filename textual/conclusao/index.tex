\chapter{Conclusão}

Tornou-se evidente através deste trabalho a vulnerabilidade do atual modelo de comunicação cliente-servidor em relação às mudanças ocorridas ao longo do tempo na especificação de APIs Web. Isso porque, com o aumento no número de clientes diversificados, surgiu uma crescente demanda em satisfazer alterações no fluxo de dados sem criar versionamento de API.

O esforço dedicado neste trabalho em resolver esta vulnerabilidade resultou na proposta de um novo modelo de comunicação possível de ser aplicado através de uma ferramenta que realiza a automação na execução de consultas de dados sobre APIs Web. Serviços que hesitam realizar mudanças na especificação devido à quebra de contrato, apresentam agora uma opção de evitar sua criação através da busca de dados em clientes através da ferramenta.

Além disso, uma das principais contribuição deste trabalho foi em mostrar os benefícios da automação na execução de consultas de dados e da facilidade na composição de serviços que a ferramenta proporciona. Por fim, mostrar a importância de escrever um código de busca através de linguagens de consulta como GraphQL e disponibilizar metadados da API.

\section[Trabalhos Futuros]{Trabalhos Futuros}

\begin{itemize}
  \item Realização de testes de validação para a composição de serviços.
  \item Criação de novos adaptadores da ferramenta para formatos de descrição de APIs. (OpenAPI, RAML, API Blueprint)
  \item Implementação da especificação da ferramenta em outras plataformas de desenvolvimento. (Mobile, Desktop, etc)
  \item Aprimoramento do algoritmo de análise de consultas.
\end{itemize}

\begin{itemize}
\item Realizar mais testes de validação como a de composição de serviços, transição entre versões de API e estilos de arquitetura.
\item Criar novos adaptadores da ferramenta para formatos de descrição de APIs. (OpenAPI, RAML, API Blueprint)
\item Implementar a especificação da ferramenta em outras plataformas de desenvolvimento. (Android, iOS, etc)
\item Melhorar o algoritmo de análise de consultas AST.
\end{itemize}

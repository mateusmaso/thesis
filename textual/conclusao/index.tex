\chapter{Conclusão}

Tornou-se evidente através deste trabalho a vulnerabilidade do atual modelo de comunicação cliente-servidor em relação à possíveis mudanças no fluxo de dados ocorridas ao longo do tempo sobre APIs Web. Isso porque, com o aumento no número de serviços e diversidade de clientes distribuídos, existe uma necessidade em satisfazer a demanda de dados por clientes sem causar dependência e criação de contratos de acesso.

O esforço dedicado neste trabalho em resolver esta vulnerabilidade resultou na proposta de um novo modelo de comunicação possível de ser aplicado. Serviços que hesitam realizar mudanças no fluxo de dados devido à quebra de contrato, apresentam agora uma opção de evitar sua criação através da descrição dos metadados de sua API e o uso da ferramenta proposta para a busca de dados em seus clientes.

Além disso, uma das principais contribuição deste trabalho foi em mostrar os benefícios da busca de dados automatizada e a facilidade na composição de serviços que a ferramenta proporciona em clientes. Em seguida, mostrar a importância de escrever código de busca através de expressões como consultas GraphQL, ao invés de chamadas diretas em pontos de acesso de uma API.

Por fim, o trabalho é um exemplo de oportunidade concretizada graças ao uso formatos para descrição de APIs Web.

\section[Trabalhos Futuros]{Trabalhos Futuros}

\begin{itemize}
  \item Realização de testes de validação para a composição de serviços.
  \item Criação de novos adaptadores da ferramenta para formatos de descrição de APIs. (OpenAPI, RAML, API Blueprint)
  \item Implementação da especificação da ferramenta em outras plataformas de desenvolvimento. (Mobile, Desktop, etc)
  \item Aprimoramento do algoritmo de análise de consultas.
\end{itemize}

\begin{itemize}
\item Realizar mais testes de validação como a de composição de serviços, transição entre versões de API e estilos de arquitetura.
\item Criar novos adaptadores para formatos de descrição de APIs. (OpenAPI, RAML, API Blueprint)
\item Implementar a especificação da ferramenta em outras plataformas de desenvolvimento. (Android e iOS)
\item Melhorar o algoritmo de análise de consultas AST.
\end{itemize}

\section[JSON Schema]{JSON Schema}

JSON Schema é uma linguagem de definição projetada para descrever estruturas de dados JSON por meio de esquemas. Proposta em 2009 por Kris Zyp, teve como objetivo fornecer um contrato para que aplicações soubessem como trabalhar e interagir com estruturas de dados. Por meio deste, é possível prever representações e assim realizar operações de validação, documentação, navegação por \textit{hyperlinks} e controle de iteração. \cite{Zyp2013}

Por ser uma linguagem de simples uso, para modelar um esquema basta construir um objeto JSON utilizando um subconjunto válido de chaves especias descritas pela linguagem. No entanto, funcionalidades como descrição de estruturas multimídia\footnote{
 Capaz de reunir diversas mídias, como imagens, textos, vídeos e audio digital, por exemplo.
} e a navegação de dados são apenas disponibilizadas no formato JSON Hyper-Schema, uma variação da linguagem de especificação. \cite{Jackson2016}

\begin{figure}[H]
  \centering
  \begin{minted}[frame=single,framesep=10pt,fontsize=\footnotesize]{text}
    {
      "\$schema": "http://json-schema.org/draft-04/schema#",
      "title": "Pessoa",
      "description": "Uma pessoa",
      "type": "object",
      "required": ["nome", "aniversario"],
      "properties": {
        "nome": {
          "type": "string"
        },
        "aniversario": {
          "type": "string"
        },
        "cidade": {
          "type": "string"
        }
      }
    }
  \end{minted}
  \caption{JSON Schema para Figura 3}
\end{figure}

\begin{figure}[H]
  \centering
  \begin{minted}[frame=single,framesep=10pt,fontsize=\footnotesize]{text}
    {
      "\$schema": "http://json-schema.org/draft-04/hyper-schema#",
      ...
      "properties": {
        ...
        "foto": {
          "media": {
            "binaryEncoding": "base64",
            "type": "image/png"
          }
        }
      },
      "links": [
        {
          "rel": "foto",
          "href": "/{id}.png",
          "mediaType": "image/png"
        }
      ]
    }
  \end{minted}
  \caption{JSON Hyper-Schema para Figura 3}
\end{figure}

Ao exemplo das figuras 5 e 6, ambos os esquemas asseguram que, dada uma estrutura JSON, para que esta seja reconhecida como uma entidade “pessoa”, deve conter as propriedades "nome" e "aniversario" com valores do tipo "string". Já na figura 6, além das estruturas definidas pela figura 5, é descrita uma nova propriedade "foto" do tipo multimídia, além de como navegar em busca desta informação.

Em casos onde a complexidade de um esquema começa a crescer, é comum a definição de sub-esquemas através da chave “definitions”. Desta forma, podem ser referenciadas pela chave "\$ref" permitindo o reuso de estruturas dentro de um esquema. Vale lembrar que a chave “definitions” é apenas um mecanismo útil para definir esquemas em um lugar comum, entretanto, não sugerem que estas propriedades sejam validadas em um objeto ao menos que referenciadas em outras estruturas do esquema. \cite{Leach2014}

\begin{figure}[H]
  \centering
  \begin{minted}[frame=single,framesep=10pt,fontsize=\footnotesize]{text}
    {
      "\$schema": "http://json-schema.org/draft-04/hyper-schema#",
      ...
      "definitions": {
        "cidade": {
          "type": "string",
          "properties": {
            "nome": { "type": "string" },
            "estado": { "type": "string" },
            "pais": { "type": "string" }
          }
        }
      },
      "properties": {
        ...
        "cidade": {
          "\$ref": "#/definitions/cidade"
        }
      },
      "links": [
        ...
        {
          "rel": "cidade",
          "href": "/{id}/cidade",
          "targetSchema": {
            "\$ref": "#/definitions/cidade"
          }
        }
      ]
    }
  \end{minted}
  \caption{JSON Hyper-Schema para Figura 4 usando \$ref}
\end{figure}

Como boa prática, é recomendado (mas não necessário) o uso da chave especial “\$schema” para determinar quando uma estrutura JSON está sendo representada em forma de esquema. Na tabela 3 são descritas algumas das chaves especiais usadas para descrever objetos em esquemas. \cite{Droettboom2015}

\begin{table}[ht!]
  \centering
  \begin{tabular}{|c|c|}
    \hline
    Chave & Descrição \\
    \hline
    \$schema & Identificador de versão \\
    \hline
    type & Tipo de dado \\
    \hline
    title & Nome da estrutura \\
    \hline
    description & Propósito da estrutura \\
    \hline
    required & Lista de propriedades com presença obrigatória \\
    \hline
    properties & Propriedades usadas para validar uma estrutura \\
    \hline
    definitions & Propriedades (sub-esquemas) para referência \\
    \hline
    ... & ... \\
    \hline
    links & Lista de Link Description Objects (LDO) \\
    \hline
  \end{tabular}
  \caption{Subconjunto de chaves especiais JSON Schema}
\end{table}

De certa forma, JSON Schema continua sendo uma das únicas tentativas sérias de propor uma linguagem de definição para o formato JSON. Contudo, ainda está longe de ser considerada padrão, mas já há um número crescente de aplicações que suportam o formato, além de uma quantidade significativa de ferramentas que permitem sua validação. \cite{PezoaEtAl2016}

Vale lembrar também que, segundo Leach, com o recente surgimento de grandes formatos de descrição de APIs ao longo dos últimos anos, JSON Hyper-Schema tem-se tornado uma ótima opção para descrever, não apenas estruturas de requisição e resposta JSON mas também como documentar e navegar por pontos de acesso de APIs. \cite{Leach2014}

Nas próximas seções será abordado um dos estilos de arquitetura mais utilizados hoje em dia em APIs Web, além de soluções encontradas no mercado para descrição destes serviços.

\section[Serialização de Dados]{Serialização de Dados}

Na ciência da computação, serialização de dados é um processo de tradução usado para converter estruturas de dados\footnote{
  Uma estrutura de dados é uma forma abstrata de representar e organizar dados. Seu objetivo é ajudar a reduzir complexidade, podendo armazenar dados de diferentes tipos, como números, strings ou até mesmo outras estruturas de dados.
} em formatos que possam ser armazenados, transmitidos e reconstruídos por um mesmo ou por outro ambiente computacional. \cite{Cline2016}

Dados serializados normalmente têm um tempo de vida maior que suas aplicações de origem e, ao serem armazenados em disco ou transmitidos pela rede, possuem representação diferente do que sua estrutura em memória. Para se ler dados serializados em memória é preciso realizar o processo inverso, também chamado de desserialização, onde estes passam a ser representados por estruturas da linguagem de execução. \cite{Guller2016}
  
\begin{figure}[H]
  \centering
  \smartdiagramset{circular distance=2cm,uniform arrow color=true,arrow line width=1pt,arrow color=gray!50!black,uniform color list=white for 6 items,font=\small,text width=2.0cm,module minimum width=2.0cm,module minimum height=1.5cm,arrow tip=to}
  \smartdiagram[flow diagram:horizontal]{Estrutura de dados,Serialização,Estrutura serializada,Desserialização}
  \caption{Processo de serialização e desserialização}
\end{figure}

Este processo, embora demande tempo, permitiu que aplicações fizessem o consumo de informações de forma distribuída, contribuindo com o aumento do volume de dados que circulam pela internet. Além disso, fez-se necessária a seleção adequada de formatos de serialização cuja estrutura não prejudicasse o desempenho de aplicações na busca por dados. \cite{SumarayMakki2012}

Segundo a Cisco Systems, de 2014 para 2015, houve um aumento de 21\% no volume de tráfego de dados registrados apenas por seus aparelhos. As categoria Web, Email e Data foram responsáveis por representar aproximadamente 7,558 petabytes de dados transmitidos por seus clientes durante um mês. \cite{Cisco2016}

Para suprir esta alta demanda, diversos formatos de serialização foram introduzidos para melhor atender os problemas de desempenho experienciados por serviços. Dentre eles, tempo de serialização e desserialização, tamanho de transferência, flexibilidade de uso, facilidade de leitura, automação, suporte para linguagem, entre outros. \cite{Guller2016}

\begin{table}[ht!]
  \centering
  \resizebox{\columnwidth}{!}{
    \begin{tabular}{|c|c|c|c|c|}
      \hline
      Formato & Especificação & Codificação & Human-Readable & Esquema/IDL \\
      \hline
      XML & Padronizada & Textual & Sim & Sim \\
      \hline
      JSON & Padronizada & Textual & Sim & Parcial \\
      \hline
      YAML & Padronizada & Textual & Sim & Parcial \\
      \hline
      Avro & Padronizada & Binário & Não & Sim \\
      \hline
      P. Buffers & Padronizada & Binário & Parcial & Sim \\
      \hline
      Thrift & Não Padronizada & Binário & Parcial & Sim \\
      \hline
    \end{tabular}
  }
  \caption{Comparação de formatos de serialização}
\end{table}

Para melhor entender cada formato, será feita uma abordagem sobre algumas das categorias de classificação utilizadas para estudar o desempenho dos formatos existentes hoje em dia. \\

\textbf{Especificação} \\

Um formato pode ter sua especificação classificada como padronizada ou não padronizada. Uma especificação padronizada é regida por requisitos que auxiliam na reprodutibilidade do processo em outras linguagens para maximização da compatibilidade e minimização de erros. Ao contrário, dada uma linguagem de programação, não é garantido que sua implementação esteja seguindo os padrões e poderá ser considerado como não padronizada. \cite{McDermid1991} \\

\textbf{Codificação} \\

Codificação é o processo de sequenciamento de caracteres usado na redução da transmissão e armazenamento de dados. É possível classificar em dois tipos: textuais e binários. Formatos baseados em texto não são codificados para seu fácil acesso e podem ser lidos diretamente através de editores de texto. Já um formato binário faz o uso intensivo da codificação e decodificação para salvar espaço. \cite{Queiros2014} \\

\textbf{Human-Readable} \\

Ao representar estruturas de dados em formatos de serialização para que máquinas possam fazer a leitura, não é garantido, no entanto, que esta representação também seja legível por seres humanos.

Para que um formato seja human-readable, além de máquinas, pessoas devem conseguir ler e dizer o que está sendo representado na estrutura, mesmo fora de contexto. Para desenvolvedores, este detalhe é essencial no processo de debugging\footnote{
  Depuração é o processo de encontrar e reduzir defeitos num aplicativo de software ou mesmo em hardware.
}. Nota-se que a leitura de um formato é diferente de seu entendimento, uma vez que nem todos os formatos possuem maneiras de descrever seus metadados. O formato JSON, por exemplo, é baseado em texto e tem como objetivo a facilidade de uso e legibilidade por desenvolvedores. Nem sempre, contudo, é possível identificar o que está sendo representado em suas estruturas. \cite{SumarayMakki2012} \\

\textbf{Esquema/IDL} \\

Com objetivo de entender o que está sendo representado, alguns formatos disponibilizam na sua especificação maneiras de descrever metadados. Esta categoria é importante principalmente para que máquinas consigam inferir quais estruturas estão sendo lidas e, assim, tomar decisões de forma autônoma.

Um formato descritivo normalmente disponibiliza estruturas como esquemas IDL\footnote{
  Linguagem de descrição utilizada para descrever a interface dos componentes de um software.
} para descrição da própria representação. À medida que estas descrições são incorporadas dentro da mesma representação, é possível classificar estes formatos como sendo auto-descritivos. \cite{Rentachintala2014}

\section{JSON}

JSON ou Javascript Object Notation é um formato de serialização de dados human-readable baseado em texto com especificação padronizada e parcialmente descritivo. Foi desenvolvido por Douglas Crockford com o objetivo de representar dados em uma maneira simples, leve e flexível através da redução na sobrecarga de marcações comparado ao formato XML.

Por ter se adaptado bem no ambiente de aplicações distribuídas, este formato acabou sendo amplamente utilizado em serviços como principal forma de representação de dados serializados. \cite{Duvander2013}

Na sua essência, o JSON foi construído com base em quatro tipos primitivos de dados e outros dois para composição. Cada tipo possui seu respectivo correspondente na maioria das linguagens de programação, embora possam ser identificados por nomes diferentes. \cite{Droettboom2015}

\begin{table}[H]
  \centering
  \begin{tabular}{|c|c|c|c|c|}
    \hline
    Tipo & Exemplo de Valor \\
    \hline
    object & \mintinline[fontsize=\footnotesize]{text}{ {"chave1": "valor1", "chave2": "valor2"} } \\
    \hline
    array & \mintinline[fontsize=\footnotesize]{text}{ ["primeiro", "segundo", "terceiro"] } \\
    \hline
    number & \mintinline[fontsize=\footnotesize]{text}{ 1, -1, 2.9999 } \\
    \hline
    string & \mintinline[fontsize=\footnotesize]{text}{ "Isso é uma string" } \\
    \hline
    boolean & \mintinline[fontsize=\footnotesize]{text}{ true, false } \\
    \hline
    null & \mintinline[fontsize=\footnotesize]{text}{ null } \\
    \hline
  \end{tabular}
  \caption{Tipos de valores em JSON}
\end{table}

Através da composição de listas, objetos e tipos primitivos, consegue-se representar complexas estruturas de dados. Não existe, no entanto, um único padrão de representação. Dada uma estrutura, é possível representá-la de inúmeras maneiras. A seguir estão duas formas diferentes de representação em JSON de uma entidade “pessoa”:
 \cite{Droettboom2015}

\begin{figure}[H]
  \centering
  \begin{minted}[frame=single,framesep=10pt,fontsize=\footnotesize]{text}
    {
      "nome": "Mateus Maso",
      "aniversario": "25 de março de 1992",
      "cidade": "Florianópolis, SC, Brasil"
    }
  \end{minted}
  \caption{Primeiro exemplo de representação JSON}
\end{figure}

\begin{figure}[H]
  \centering
  \begin{minted}[frame=single,framesep=10pt,fontsize=\footnotesize]{text}
    {
      "nome": "Mateus",
      "sobrenome": "Maso",
      "nascimento": "25-03-1992",
      "cidade": {
        "nome": "Florianópolis",
        "estado": "SC",
        "pais": "Brasil"
      }
    }
  \end{minted}
  \caption{Segundo exemplo de representação JSON}
\end{figure}

Ambas representações são válidas, apesar da figura 4 estar representando os dados em uma estrutura um pouco mais formal. No entanto, por ser um formato não descritivo, a responsabilidade de entender o que está sendo representado vai depender da análise crítica ou conhecimento prévio dos desenvolvedores. Já uma máquina, sem conhecer o contexto, não saberia como interpretar os dados de forma correta. \cite{Droettboom2015}

Para resolver isso, será abordado em seguida um dos formatos existentes hoje em dia para a descrição de estruturas JSON.

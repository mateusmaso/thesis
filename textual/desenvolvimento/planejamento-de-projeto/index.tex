\section{Planejamento de Projeto}

Para que clientes possam trabalhar com dados remotos, é preciso que haja um meio de buscá-los antes de executar qualquer lógica que dependa deles. Para isso, a solução comumente adotada é a implementação de um código de busca para acesso remoto de dados em APIs de serviços\footnote{
  Infraestrutura distribuída (servidores, banco de dados, etc) que respondem a pedidos de operações oriundas de clientes em forma de requisições de API.
}.

Para que se possa manter garantia na comunicação, ao passar do tempo, fez-se necessário estabelecer um contrato de acesso entre o cliente e a interface do serviço. Isso porque a atual implementação do código de busca por clientes não prevê mudanças na especificação da API. Este contrato é, portanto, representado no modelo de comunicação da figura 15 através de chamadas entre o código de busca do cliente e a API do serviço.

\begin{figure}[H]
  \centering
  \begin{tikzpicture}[font=\small]
    \node (client) at (-3,0) {\includegraphics[width=1.0cm]{figuras/client}};
    \node[right of=client] (clientFetchCode) at (-2.2,0.7) {\includegraphics[width=1.1cm]{figuras/code}};
    \node[below of=clientFetchCode, node distance=1.6cm] (clientLogicCode) {\includegraphics[width=1.1cm]{figuras/code}};
    \node[rectangle,minimum width=0.8cm, minimum height=3cm,draw,right of=clientFetchCode, node distance=3.0cm] (api) {$API$};
    \node[right of=api, node distance=1.8cm] (server) {\includegraphics[width=1.0cm]{figuras/server}};
    \node[below of=server, node distance=1.8cm] (datastore) {\includegraphics[width=1.0cm]{figuras/database}};
    \node[above of=server, node distance=1.8cm] (data) {\includegraphics[width=1.1cm]{figuras/code}};
    \node[cloud, cloud puffs=30, minimum width=11cm, minimum height=10cm, draw,style={scale=0.6}] (service) at (server) {};
    \node[below of=clientFetchCode,node distance=0.0cm] {\textless$fetch$\textgreater};
    \node[below of=clientLogicCode,node distance=0.0cm] {\textless$logic$\textgreater};
    \node[below of=data,node distance=0.0cm] {\{$data$\}};
    \node[below of=service,node distance=3.6cm] {$Service$};
    \node[below of=client,node distance=1.8cm] {\ldots};
    \node[right of=server,node distance=1.8cm] {\ldots};
    \draw[decorate,decoration={brace,raise=0.2cm,mirror}] ([yshift=6pt]clientFetchCode.north west) -- ([yshift=-12pt]clientLogicCode.south west);
    \draw [->] ([yshift=0.25cm]clientFetchCode.east) -- ([yshift=0.25cm]api.west);
    \draw [->] (api) -- (clientFetchCode);
  \end{tikzpicture}
  \caption{Modelo de comunicação entre cliente e serviço}
\end{figure}

O modelo de comunicação representado pode não revelar problemas para serviços com pouca demanda de acesso e diversidade de consultas. Contudo, ao longo do tempo e com o aumento na quantidade de contratos, alterações na especificação como, por exemplo, as de fluxo de dados\footnote{
  Fluxo de acesso por clientes para consulta de dados sobre APIs de serviços.
} podem se tornar um desafio.

Mudanças no fluxo de dados pela API são inevitáveis em aplicações distribuídas. Elas ocorrem para abraçar a constante transformação por clientes na execução de consultas de dados. Com isso, busca-se manter uma comunicação eficiente através da redução no número de chamadas executadas na API e do tamanho de dados transmitidos pela rede. Essas mudanças podem ser desde uma simples alteração no nome de um método ou número argumentos, até as mais complexas, como dados de retorno e estilo de arquitetura. A figura 16 ilustra o fluxo de dados entre um cliente e um serviço.

\begin{figure}[H]
  \centering
  \begin{tikzpicture}[font=\small]
    \draw (-2,0) -- (-2,-5) (2,0) -- (2,-5);
    \node at (-2,0.3) {Client};
    \node at (2,0.3) {API (Service)};
    \node[loosely dotted] (virtualData) at (-4,-1){\includegraphics[width=1.1cm]{figuras/code}};
    \node[below of=virtualData,node distance=0.0cm] {\{$data$\}};
    \node (data) at (-4,-4.5) {\includegraphics[width=1.1cm]{figuras/code}};
    \node[below of=data,node distance=0.0cm] {\{$data$\}};
    \node[cloud, cloud puffs=16,draw,minimum height=1.2cm] (cloudData) at (4,-2.5) {\includegraphics[width=1.1cm]{figuras/code}};
    \node[below of=cloudData,node distance=0.0cm] {\{$data$\}};
    \draw[->] (-2,-1) -- node[midway,above] {$request$} (2,-1.5);
    \draw[<-] (-2,-2.5) -- node[midway,above] {$response$} (2,-2);
    \draw[->] (-2,-3) -- node[midway,above] {$request$} (2,-3.5);
    \draw[<-] (-2,-4.5) -- node[midway,above] {$response$} (2,-4);
  \end{tikzpicture}
  \caption{Fluxo de dados entre cliente e API}
\end{figure}

O impacto causado em clientes por mudanças no fluxo de dados de uma API felizmente é previsível, pois afeta diretamente, em sua maioria, o código de busca. Por outro lado, mudanças como a alteração de campos nas estruturas de dados, como renomear um campo "nome" para "nome\_completo", são de um grau de complexidade maior, pois pode não ter impacto no código de busca e sim na lógica da aplicação, onde é bem mais difícil de depurar erros.

Com o objetivo de viabilizar um novo modelo de comunicação a fim de evitar o impacto no código de busca em clientes devido a mudanças no fluxo de dados, fez-se necessário solucionar os seguintes problemas descritos na tabela 4. \\

\begin{table}[H]
  \begin{tabularx}{\linewidth}{>{\parskip1ex}X@{\kern4\tabcolsep}>{\parskip1ex}X}
    \toprule
    \hfil\bfseries Problema
    &
    \hfil\bfseries Solução
    \\\cmidrule(r{3\tabcolsep}){1-1}\cmidrule(l{-\tabcolsep}){2-2}

    (1) Garantir que não haja criação de contrato entre código de busca e API.\par
    (2) Realizar mudanças no fluxo de dados de uma API sem interferir no código de busca dos clientes.\par
    (3) Encontrar uma forma de expressar requisições entre o cliente e o intermediador.\par
    (4) Mapear consultas no intermediador em requisições para API de serviços.\par

    &

    (1) Construir um intermediador responsável por manter a comunicação entre cliente e serviço.\par
	(2) Evitar que clientes escrevam código de busca voltado à especificação da API, e sim para o intermediador.\par
    (3) Usar uma linguagem de consulta para expressar dependência de dados.\par
    (4) Analisar dependência de dados e formas de acesso através de metadados da API de serviços.\par

\\\bottomrule
  \end{tabularx}
  \caption{Problema-solução na concepção do novo modelo}
\end{table}

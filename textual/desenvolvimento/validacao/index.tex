\section{Validação}

Com o objetivo de validar o modelo proposto, é feito uma pesquisa comparativa focando o impacto sobre uso da ferramenta implementada em relação à mudanças no fluxo de dados. Para isso será desenvolvido um ambiente de validação onde dois clientes, um com a ferramenta e o outro não, realizam três consultas na API de um serviço REST. Após, será aplicado uma série de mudanças no fluxo de dados da API e coletado os dados para análise do impacto causado no código de busca desses clientes. \\

\textbf{Escopo de serviço} \\

Foi escolhido trabalhar com um escopo de serviço conhecido como SWAPI (\textit{Starwars API}), bastante utilizado para análise de implementações de tecnologias e estilos de arquitetura em linguagens de programação. Nele, são descritos seis entidades (Pessoa, Filme, Espaçonave, Veículo, Espécie, Planeta) e seus respectivos relacionamentos. Para o escopo do projeto, será implementado uma API REST que expõe consultas dessas entidades em formato JSON através de pontos de acesso em URIs, visto na tabela 6. Nota-se que as representações possuem apenas um nível de expansão, ou seja, para acesso dos relacionamentos é preciso consultar a API através de links descritos pela estrutura de retorno (HATEOAS).

\begin{table}[H]
  \centering
  \begin{tabular}{|l|c|}
    \hline
    URI & Descrição \\
    \hline
    /people & Busca lista de pessoas \\
    \hline
    /people/:id & Busca pessoa pelo id \\
    \hline
    /films & Busca lista de filmes \\
    \hline
    /films/:id & Busca filme pelo id \\
    \hline
    /starships & Busca lista de espaçonaves \\
    \hline
    /starships/:id & Busca espaçonave pelo id \\
    \hline
    /vehicles & Busca lista de veículos \\
    \hline
    /vehicles/:id & Busca veículo pelo id \\
    \hline
    /species & Busca lista de espécies \\
    \hline
    /species/:id & Busca espécie pelo id \\
    \hline
    /planets & Busca lista de planetas \\
    \hline
    /planets/:id & Busca planeta pelo id \\
    \hline
  \end{tabular}
  \caption{Pontos de acesso (GET) SWAPI}
\end{table}

\textbf{Perguntas e respostas esperadas} \\

Com o propósito de explorar cada ponto de acesso para busca de dados na SWAPI, foram pensadas em três perguntas de média complexidade, onde fossem envolvidos ao menos três das entidades do escopo. Para cada pergunta existe apenas uma resposta certa, onde sua lógica é baseada em campos das estruturas de dados de retorno.

\begin{enumerate}
\item[\textbf{Q1.}] Qual o nome do filme no qual aparece mais personagens oriundos de um planeta deserto? \textbf{R:} "Revenge of the Sith"
\item[\textbf{Q2.}] Qual o nome da espécie predominante entre os habitantes do planeta "Tatooine"? \textbf{R:} "Droid"
\item[\textbf{Q3.}] Qual o nome do personagem que mais pilota espaçonaves e veículos durante o filme "A New Hope"? \textbf{R:} "Chewbacca"
\end{enumerate}

No intuito de acertar as respostas, a seguir na tabela 7, é descrito o fluxo de dados necessário de cada pergunta para a busca de dados por clientes.

\begin{table}[H]
  \centering
  \begin{tabular}{|c|c|c|}
    \hline
    Pergunta & Fluxo de Dados & Número de requisições \\
    \hline
    Q1 & \begin{minipage}[t]{0.3\textwidth}
      \begin{itemize}
        \item[\textbf{GET}] /api/films
        \item[\textbf{GET}] /api/people/:id
        \item[\textbf{GET}] /api/planet/:id
      \end{itemize}
    \end{minipage} & \begin{minipage}[t]{0.5\textwidth}
      \begin{itemize}
        \item[\textbf{x1}] films
        \item[\textbf{x162}] films.characters
        \item[\textbf{x162}] films.characters.homeworld
      \end{itemize}
    \end{minipage} \\
    \hline
    Q2 & \begin{minipage}[t]{0.3\textwidth}
      \begin{itemize}
        \item[\textbf{GET}] /api/planets/1
        \item[\textbf{GET}] /api/people/:id
        \item[\textbf{GET}] /api/species/:id
      \end{itemize}
    \end{minipage} & \begin{minipage}[t]{0.5\textwidth}
      \begin{itemize}
        \item[\textbf{x1}] planet
        \item[\textbf{x10}] planet.residents
        \item[\textbf{x2}] planet.residents.species
      \end{itemize}
    \end{minipage} \\
    \hline
    Q3 & \begin{minipage}[t]{0.3\textwidth}
      \begin{itemize}
        \item[\textbf{GET}] /api/films/1
        \item[\textbf{GET}] /api/starships/:id
        \item[\textbf{GET}] /api/people/:id
        \item[\textbf{GET}] /api/vehicles/:id
        \item[\textbf{GET}] /api/people/:id
      \end{itemize}
    \end{minipage} & \begin{minipage}[t]{0.5\textwidth}
      \begin{itemize}
        \item[\textbf{x1}] film
        \item[\textbf{x8}] film.starships
        \item[\textbf{x9}] film.starships.pilots
        \item[\textbf{x4}] film.vehicles
        \item[\textbf{x0}] film.vehicles.pilots
      \end{itemize}
    \end{minipage} \\
    \hline
  \end{tabular}
  \caption{Fluxo de dados para responder as perguntas}
\end{table}

\textbf{Mudanças no fluxo de dados} \\

Para avaliar o impacto no código dos dois clientes, é proposto quatro tipos de mudanças não acumulativas no fluxo de dados. Cada mudança busca testar a capacidade da ferramenta em se adaptar e realizar a comunicação mesmo após a alteração no fluxo. Nota-se que, para cada mudança na API do serviço, é preciso a atualização de seus metadados para que a ferramenta consiga operar direito. Na tabela 8, é descrito o changelog das mudanças realizadas.

\begin{table}[H]
  \centering
  \begin{tabular}{|c|c|c|}
    \hline
    Mudança & Descrição & Changelog \\
    \hline
    C1 & Mudança no endereço de ponto de acesso. & \begin{minipage}[t]{0.8\textwidth}
      \begin{itemize}
        \item Renomeação da URI /films para /movies.
        \item Renomeação da URI /films/:id para /movies/:id
      \end{itemize}
    \end{minipage} \\
    \hline
    C2 & Mudança no nível de estrutura de resposta. & \begin{minipage}[t]{0.8\textwidth}
      \begin{itemize}
        \item Expansão do campo pilots da URI /starships/:id
        \item Expansão do campo pilots da URI /vehicles/:id
      \end{itemize}
    \end{minipage} \\
    \hline
    C3 & Adição de ponto de acesso otimizado. & \begin{minipage}[t]{0.8\textwidth}
      \begin{itemize}
        \item[\textbf{+}] Adição da URI /tatooine.
        \item[\textbf{+}] Adição da URI /films/:id/characters.
      \end{itemize}
    \end{minipage} \\
    \hline
    C4 & Remoção de ponto de acesso deprecado. & \begin{minipage}[t]{0.8\textwidth}
      \begin{itemize}
        \item[\textbf{+}] Adição da URI /people/:id/homeworld.
        \item[\textbf{$-$}] Remoção da URI /planet/:id.
      \end{itemize}
    \end{minipage} \\
    \hline
  \end{tabular}
  \caption{Changelog do novo fluxo de dado para cada mudança}
\end{table}

% \begin{table}[H]
%   \centering
%   \begin{tabular}{|c|c|c|c|}
%     \hline
%     Mudança & Q1 & Q2 & Q3 \\
%     \hline
%     C1 & \begin{minipage}[t]{0.3\textwidth}
%       \begin{itemize}
%         \item[\textbf{GET}] /api/movies
%         \item[\textbf{GET}] /api/people/:id
%         \item[\textbf{GET}] /api/planet/:id
%       \end{itemize}
%     \end{minipage} & \begin{minipage}[t]{0.3\textwidth}
%       \begin{itemize}
%         \item[\textbf{GET}] /api/planets/1
%         \item[\textbf{GET}] /api/people/:id
%         \item[\textbf{GET}] /api/species/:id
%       \end{itemize}
%     \end{minipage} & \begin{minipage}[t]{0.3\textwidth}
%       \begin{itemize}
%         \item[\textbf{GET}] /api/movies/1
%         \item[\textbf{GET}] /api/starships/:id
%         \item[\textbf{GET}] /api/people/:id
%         \item[\textbf{GET}] /api/vehicles/:id
%         \item[\textbf{GET}] /api/people/:id
%       \end{itemize}
%     \end{minipage} \\
%     \hline
%     C2 & \begin{minipage}[t]{0.3\textwidth}
%       \begin{itemize}
%         \item[\textbf{GET}] /api/films
%         \item[\textbf{GET}] /api/people/:id
%         \item[\textbf{GET}] /api/planet/:id
%       \end{itemize}
%     \end{minipage} & \begin{minipage}[t]{0.3\textwidth}
%       \begin{itemize}
%         \item[\textbf{GET}] /api/planets/1
%         \item[\textbf{GET}] /api/people/:id
%         \item[\textbf{GET}] /api/species/:id
%       \end{itemize}
%     \end{minipage} & \begin{minipage}[t]{0.3\textwidth}
%       \begin{itemize}
%         \item[\textbf{GET}] /api/films/1
%         \item[\textbf{GET}] /api/starships/:id
%         \item[\textbf{GET}] /api/vehicles/:id
%       \end{itemize}
%     \end{minipage} \\
%     \hline
%     C3 & \begin{minipage}[t]{0.3\textwidth}
%       \begin{itemize}
%         \item[\textbf{GET}] /api/films
%         \item[\textbf{GET}] /api/films/:id/characters
%         \item[\textbf{GET}] /api/planet/:id
%       \end{itemize}
%     \end{minipage} & \begin{minipage}[t]{0.3\textwidth}
%       \begin{itemize}
%         \item[\textbf{GET}] /api/tatooine
%         \item[\textbf{GET}] /api/species/:id
%       \end{itemize}
%     \end{minipage} & \begin{minipage}[t]{0.3\textwidth}
%       \begin{itemize}
%         \item[\textbf{GET}] /api/films/1
%         \item[\textbf{GET}] /api/starships/:id
%         \item[\textbf{GET}] /api/people/:id
%         \item[\textbf{GET}] /api/vehicles/:id
%         \item[\textbf{GET}] /api/people/:id
%       \end{itemize}
%     \end{minipage} \\
%     \hline
%     C4 & \begin{minipage}[t]{0.3\textwidth}
%       \begin{itemize}
%         \item[\textbf{GET}] /api/films
%         \item[\textbf{GET}] /api/people/:id
%         \item[\textbf{GET}] /api/people/:id/homeworld
%       \end{itemize}
%     \end{minipage} & - & \begin{minipage}[t]{0.3\textwidth}
%       \begin{itemize}
%         \item[\textbf{GET}] /api/films/1
%         \item[\textbf{GET}] /api/starships/:id
%         \item[\textbf{GET}] /api/people/:id
%         \item[\textbf{GET}] /api/vehicles/:id
%         \item[\textbf{GET}] /api/people/:id
%       \end{itemize}
%     \end{minipage} \\
%     \hline
%   \end{tabular}
%   \caption{Novo fluxo de dados para cada mudança}
% \end{table}

\textbf{Consultas no intermediador} \\

\begin{table}[H]
  \centering
  \begin{tabular}{|c|c|}
    \hline
    Pergunta & Consulta GraphQL \\
    \hline
    Q1 & \begin{minipage}[t]{0.5\textwidth}
      \begin{minted}{text}
      {
        allFilms {
          title
          characters {
            homeworld {
              climate
            }
          }
        }
      }
      \end{minted}
    \end{minipage} \\
    \hline
    Q2 & \begin{minipage}[t]{0.5\textwidth}
      \begin{minted}{text}
      {
        planet(planetID: 1) {
          residents {
            species {
              name
            }
          }
        }
      }
      \end{minted}
    \end{minipage} \\
    \hline
    Q3 & \begin{minipage}[t]{0.5\textwidth}
      \begin{minted}{text}
      {
        film(filmID: 1) {
          starships {
            pilots {
              name
            }
          }
          vehicles {
            pilots {
              name
            }
          }
        }
      }
      \end{minted}
    \end{minipage} \\
    \hline
  \end{tabular}
  \caption{Consultas GraphQL para as perguntas}
\end{table}

\textbf{Variáveis} \\

É descrito cinco variáveis para análise dos testes de validação. Todos são quantitativas e coletadas no cliente após cada processo de mudança da API. Totalizando um número de 20 dados possíveis para análise no final de sua execução.

\begin{table}[H]
  \centering
  \begin{tabular}{|c|c|c|}
    \hline
    Variável & Unidade & Tipo \\
    \hline
    Porcentagem de acerto & \% & Quantitativa \\
    \hline
    Tamanho de resposta & kb & Quantitativa \\
    \hline
    Número de requisições & inteiro & Quantitativa \\
    \hline
    Tempo de busca de metadados & ms & Quantitativa \\
    \hline
    Tempo de processamento & ms & Quantitativa \\
    \hline
  \end{tabular}
  \caption{Variáveis de coleta e análise}
\end{table}

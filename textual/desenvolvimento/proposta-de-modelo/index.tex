\section{Proposta de Modelo}

Um novo modelo é proposto com o intuito de melhorar a comunicação cliente-servidor apresentada. Observado na figura 18, este prevê a criação de uma ferramenta no cliente para a intermediação da comunicação entre o código de busca e a API de serviços. Além disso, há a necessidade de reimplementação do código de busca para uma nova linguagem de consulta que seja interpretada pelo intermediador. Da mesma forma, soma-se a criação de um arquivo no serviço para descrição de metadados da API e outro no cliente para configuração, ambos essenciais para o funcionamento da ferramenta.

\begin{figure}[H]
  \centering
  \begin{tikzpicture}[font=\small]
    \node (client) at (-3,0) {\includegraphics[width=1.0cm]{figuras/client}};
    \node[right of=client] (clientFetchCode) at (-2.2,0) {\includegraphics[width=1.1cm]{figuras/code}};
    \node[below of=clientFetchCode, node distance=1.6cm] (clientLogicCode) {\includegraphics[width=1.1cm]{figuras/code}};
    \node[above of=clientFetchCode, node distance=1.6cm] (clientConfigCode) {\includegraphics[width=1.1cm]{figuras/code}};
    \node[rectangle,draw,right of=clientFetchCode,minimum height=1cm,minimum width=1.5cm,node distance=2.0cm] (tool) {$Tool$};
    \node[rectangle,minimum width=0.8cm, minimum height=3cm,draw,node distance=1.8cm,right of=tool,node distance=2.7cm] (api) {$API$};
    \node[right of=api, node distance=1.8cm] (data) {\includegraphics[width=1.1cm]{figuras/code}};
    \node[below of=data,node distance=1.8cm] (server) {\includegraphics[width=1.0cm]{figuras/server}};
    \node[above of=data, node distance=1.8cm] (metadata) {\includegraphics[width=1.1cm]{figuras/code}};
    \node[cloud, cloud puffs=30, minimum width=11cm, minimum height=10cm, draw,style={scale=0.6}] (service) at (data) {};
    \node[below of=clientFetchCode,node distance=0.0cm] {\textless$fetch$\textgreater};
    \node[below of=clientLogicCode,node distance=0.0cm] {\textless$logic$\textgreater};
    \node[below of=clientConfigCode,node distance=0.0cm] {\textless$config$\textgreater};
    \node[below of=metadata,node distance=0.0cm] {\{$metadata$\}};
    \node[below of=data,node distance=0.0cm] {\{$data$\}};
    \node[below of=service,node distance=3.6cm] {$Service$};
    \node[below of=client,node distance=1.8cm] {\vdots};
    \node[right of=data,node distance=1.8cm] {\ldots};
    \draw[decorate,decoration={brace,raise=0.2cm,mirror}] ([yshift=6pt]clientConfigCode.north west) -- ([yshift=-12pt]clientLogicCode.south west);
    \draw [->] ([yshift=0.25cm]clientFetchCode.east) -- ([yshift=0.25cm]tool.west);
    \draw [->] (tool) -- (clientFetchCode);
    \draw [->] ([yshift=0.25cm]tool.east) -- ([yshift=0.25cm]api.west);
    \draw [->] (api) -- (tool);
    \draw [->,dashed] (clientConfigCode) -| ([xshift=-0.25cm]tool.north);
    \draw [->,dashed] (metadata) -| ([xshift=0.25cm]tool.north);
  \end{tikzpicture}
  \caption{Modelo proposto para evitar contrato de comunicação}
\end{figure}

A proposta de eliminação de contrato visa também automatizar o acesso de clientes em APIs. Através da implementação de algoritmos na ferramenta, escolhe-se o caminho de acesso de melhor custo-benefício em relação aos serviços disponíveis para o cliente. O modelo proporciona, além disso, um ambiente escalável para consulta de dados através da composição de serviços, visto na figura 19.

\begin{figure}[H]
  \centering
  \begin{tikzpicture}[font=\small]
    \node (client1) at (-2,0) {\includegraphics[width=1.0cm]{figuras/client}};
    \node (client2) at (2,0) {\includegraphics[width=1.0cm]{figuras/client}};
    \node[rectangle,draw,minimum height=1cm,minimum width=1.5cm,below of=client2,node distance=2.3cm] (tool) {$Tool$};
    \node[cloud, cloud puffs=16, draw] (service1) at (-3,-6) {$Service_1$};
    \node (serviceDots) at (0,-6) {\ldots};
    \node[cloud, cloud puffs=16,minimum height=1cm, draw] (serviceN) at (3,-6) {$Service_N$};
    \draw [->] (client1) -- (service1) node [midway,above,font=\footnotesize] {$request$};
    \draw [->] (client1) -- (serviceN) node [midway,above,font=\footnotesize] {$request$};
    \draw [->] (client2) -- (tool) node [midway,above,font=\footnotesize] {$query$};
    \draw [->] (tool) -- (service1) node [midway,above,font=\footnotesize] {$request$};
    \draw [->] (tool) -- (serviceN) node [midway,above,font=\footnotesize] {$request$};
  \end{tikzpicture}
  \caption{Composição de serviços através da ferramenta}
\end{figure}

\section{Implementação}

A fim de aplicar a especificação da ferramenta abordada no planejamento de projeto, foi proposta a implementação de onze funções com três categorias para o uso de clientes na plataforma web. Cada uma dessas funções levantadas resolve um problema para se fazer necessário chegar no objetivo de representar os dois fluxos de execução.

Dentro do conjunto de funções, quatro delas são responsáveis por criar o intermediador; três funções analisam as consultas através do formato AST; e as outras quatro transformam as consultas em requisições na API de serviços. Somente quatro são públicas e expostas para o cliente, sendo uma delas (composeSchema) a principal para configuração/uso da ferramenta, e as outras três (buildSchema, transformAST, fetchData) para serem sobrescritas por adaptadores. 

A seguir, são descritas brevemente essas três categorias de funções, ao lado de seu objetivo e a assinatura JavaScript de cada função. \\

\textbf{Criação do esquema} \\

Consiste de quatro funções e seu principal objetivo é compor, de forma assíncrona, um esquema GraphQL unificado a partir das funcões de configuração de cada serviço.

\begin{figure}[H]
  \centering
  \begin{minted}[frame=single,framesep=10pt,fontsize=\footnotesize]{text}
  function composeSchema(
    services: [Service]
  ): Promise<GraphQLSchema>

  function buildSchema(
    metadata: JSON
  ): Promise<GraphQLSchema>

  function wrapSchema(
    schema: GraphQLSchema,
    wrapper: JSON
  ): Promise<GraphQLSchema>

  function deepExtendSchema(
    schemas: [GraphQLSchema]
  ): Promise<GraphQLSchema>
  \end{minted}
  \caption{Assinatura das funções para criação do esquema}
\end{figure}

\textbf{Análise de consulta} \\

É constituído de três funções que buscam analisar consultas GraphQL feitas no esquema gerado em formato AST. A análise descreve algoritmos que simplificam, transformam e reduzem o AST principal da consulta, quebrando-a em ASTs específicos para cada serviço.

\begin{figure}[H]
  \centering
  \begin{minted}[frame=single,framesep=10pt,fontsize=\footnotesize]{text}
  function simplifyAST(
    value: AST,
    info: JSON
  ): SimplifiedAST

  function transformAST(
    metadata: JSON,
    schema: GraphQLSchema,
    ast: SimplifiedAST
  ): SimplifiedAST

  function reduceASTs(
    rootAST: SimplifiedAST,
    asts: [SimplifiedAST]
  ): [SimplifiedAST]
  \end{minted}
  \caption{Assinatura das funções para análise de consulta}
\end{figure}

\textbf{Busca de dados} \\

Consiste as quatro últimas funções responsáveis por converter os ASTs de cada serviço e realizar as respectivas requisições para a API em busca de dados.

\begin{figure}[H]
  \centering
  \begin{minted}[frame=single,framesep=10pt,fontsize=\footnotesize]{text}
  function unwrapAST(
    ast: SimplifiedAST,
    schema: GraphQLSchema,
    wrapper: JSON
  ): SimplifiedAST

  function fetchData(
    metadata: JSON,
    ast: SimplifiedAST,
    url: String
  ): Promise<JSON>

  function wrapData(
    data: JSON,
    schema: GraphQLSchema,
    wrapper: JSON
  ): JSON
  
  function deepExtendData(
    data: [JSON]
  ): JSON
  \end{minted}
  \caption{Assinatura das funções para busca de dados}
\end{figure}

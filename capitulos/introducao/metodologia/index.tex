\section[Metodologia]{Metodologia}

A primeira etapa do trabalho foi a idealização do projeto, onde foi levantada uma série de perguntas, que culminaram em um profundo estudo para comprovar se era possível resolver o problema e de que forma a solução seria implementada. Adicionalmente, foram identificadas as tecnologias que poderiam ser usadas para ajudar no desenvolvimento da ferramenta e as formas possíveis de diminuir a curva de aprendizagem da ferramenta, consequentemente aumentando as suas chances de adoção.

Em seguida, foi pensado na criação de uma interface para a ferramenta que fosse simples de usar, rodasse em serviços GraphQL já existentes e pudesse ser integrado facilmente em APIs REST que já possuem uma forma completa de descrição de metadados.

Após isso foi feito um PoC\footnote{
  Proof of concept ou Prova de conceito
} para validar o protótipo. Em paralelo, começou o processo de escrita da monografia, reimplementação da ferramenta e preparação de um ambiente de validação que pudesse enfatizar bem os problemas reais que clientes JavaScript tem sofrido devido a este acoplamento na busca de dados. 

Por fim, com o objetivo de apresentar uma comprovação da tese, foram rodados os testes de validação, coletado os dados e apresentado os resultados analisados em forma de gráficos.
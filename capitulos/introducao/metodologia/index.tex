# Metodologia

- Problema surgiu ao tentar integrar GraphQL em uma API RESTful.
	- Precisava manter dois tipos de requisições no cliente para o mesmo recurso mas em serviços diferentes.
	- E se for possível expressar as consultas no cliente na linguagem GraphQL e transformar em requisições RESTful quando não houver implementaçao por parte do interpretador GraphQL?
- Prox etapa foi estudar se era possível realizar a ferramenta que fizesse essa transformação.
	- Por sorte, GraphQL roda no cliente web. (Relay FB)
- Estudo mais a fundo em soluções que oferecem documentação de APIs que maquinas pudessem ler. (JSON Hyper-Schema)
- Criar uma interface para a ferramenta que fosse facil de usar e acoplada ao GraphQL.
- Ideia de criar um gerador de schema contendo informações sobre os serviços e que pudesse ser executado na API padrao do GraphQL.
- Após isso foi feito a prototipação e implementação.
- Depois de funcionar começo a escrita da monografia.
- Criação dos testes para validação Prove of Concept. (POC)
- Analise dos resultados.

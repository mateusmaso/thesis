\section[Metodologia]{Metodologia}

Problema surgiu ao tentar integrar GraphQL em um cliente que fazia requisições à uma API RESTful. Cliente estava completamente acoplado ao fluxo de dados, sendo que a estrutura de dados era a mesma. Como resolver isso?

Prox etapa foi a idealização do projeto. Estudar se era possível resolver esse problema e de que forma. Como criar uma ferramenta que fizesse essa intermediação. GraphQL se tornou uma dependência essencial para o sucesso da ferramenta.

Seguido, foi pensado na criação de uma interface para a ferramenta que fosse fácil de usar, rodasse em serviços GraphQL existentes e pudesse funcionar em API's REST que possuem descrição em linguagens/formatos populares e completos.

Após isso foi feito a implementação para Proof of Concept. Depois que validado, comecei a escrita da monografia e preparação de um ambiente de validação que pudesse enfatizar problemas reais na mudança fluxo de dados entre cliente e servidor. Coletar os dados, analisar e apresentar para comprovação da tese.
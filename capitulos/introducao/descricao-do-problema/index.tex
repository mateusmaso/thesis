\section[Descrição do Problema]{Descrição do Problema}

O mercado de API's web transformou o modo de comunicação entre aplicações distribuídas para busca de informações e execução de operações na web. Não é de hoje que organizações tem se preocupado em disponibilizar API's de suas aplicações em forma de serviços para consumo próprio e/ou de terceiros. Empresas como Facebook e Netflix mostram que construir uma aplicação em forma de serviço e disponibilizar uma interface de acesso é essencial para entrar rápido no mercado de plataformas emergentes, oferecer uma melhor experiência para o usuário através do desenvolvimento de clientes nativos, além de agregar valor em seu modelo de negócio ao disponibilizar dados e operações para o uso de terceiros. (ART, 2016)

Segundo ProgrammableWeb, um dos motivos que contribuiu para o aumento do número de API's de serviços foi após a introdução em 2001 do estilo de arquitetura REST. Contudo, após 15 anos de sua introdução e diversos clientes escritos com base em seu protocolo de comunicação, REST tem-se mostrado uma solução ineficiente para lidar com o acesso de estrutura de dados por um grande número de clientes diversificado. Visto que sua interface de acesso é pré-determinada, após publicado, tornando-se difícil prever a demanda de dados e realizar mudanças no fluxo sem que haja versionamento. (ProgrammableWeb, 2016)

Por outro lado, ferramentas como GraphQL (Facebook) e Falcor (Netflix) surgem com o objetivo de resolver este problema mas encontram uma baixa adoção por não conseguirem minimizar a custosa transição que clientes desenvolvidos em REST precisam passar para fugir desta arquitetura.

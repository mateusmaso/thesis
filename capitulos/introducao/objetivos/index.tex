\section[Objetivos]{Objetivos}

O objetivo deste trabalho é o desenvolvimento de uma ferramenta que ajude na prevenção do acoplamento entre clientes e API's causado pela escrita de código voltado ao acesso direto de estruturas de dados JSON. Além disso, através da análise de metadados, poder oferecer uma maneira genérica (usando GraphQL) para busca dados independente do estilo de arquitetura e fluxo de dados implementado pela API. \\

\textbf{Objetivos específicos} \\

\begin{itemize}
\item Prevenir a mudança de código em clientes devido à alterações na implementação do fluxo de dados por uma API.
\item Facilitar a transição de código em clientes ao atualizar para uma nova versão implementado por uma API.
\item Facilitar a transição de código em clientes ao atualizar para um novo estilo de arquitetura implementado por uma API.
\item Oferecer uma maneira de composição de estruturas de dados entre mais de um serviço e interface de aplicação.
\item Promover a documentação e descrição de metadados em API's. \\
\end{itemize}

\textbf{Limites da pesquisa} \\

\begin{itemize}
\item Foco exclusivo no formato JSON para representação de dados.
\item Foco exclusivo na linguagem GraphQL para consulta de dados.
\item Foco exclusivo no estilo de arquitetura REST para realizar testes.
\item Foco exclusivo na ferramenta JSON Hyper-Schema para descrição de API's REST.
\item Foco exclusivo em validar a ferramenta através de testes para prevenir a mudança de código em clientes devido à alterações na implementação do fluxo de dados por uma API.
\end{itemize}
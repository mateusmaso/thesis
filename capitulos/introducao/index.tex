% Introdução

\chapter{INTRODUÇÃO}

As orientações aqui apresentadas
são baseadas em um conjunto de normas
elaboradas pela ABNT.
Além das normas técnicas a
Biblioteca também elaborou uma série de tutoriais
e guias que estão disponíveis na sua Homepage.
\url{http://portalbu.ufsc.br/normalizacao-de-trabalhos-2/}.

\section{OBJETIVOS}

Descrição...

\subsection{Objetivo Geral}

Descrição...

\subsection{Objetivos Específicos}

Descrição...

# Introdução

- O crescimento das “API’s”
	- Estatisticas sobre o mercado
		- numeros sobre crescimento
		- numeros sobre distribuiçao de estilos
		- numeros sobre projeções futuras
	- Por que ocorreu isso?
		- SOA is the way to go
		- “dev is shifting more and more to the client.”
		- companies want get faster into new platforms.
	- Qual problema isso está causando?
		- Clientes dependentes de tipos/estilos de serviço
	- Por que isso é ruim?
		- (Perf) Estilos de API podem estar sendo sub/super utilizadas devido a constante mudança fluxo de dados em que clientes vivem.
		- (Inovação) Investir em novos modelos/tipos/estilos requer tempo para analisar o impacto no cliente.
		- (Agile) APIs tendem a crescer verticalmente e adicionar features em serviços “monoliticos” podem atrasar o crescimento de uma organização.
	- Como resolver isso?
		- Impedir que o cliente tenha acesso direto as APIs
	- De que forma?
		- Criar uma ferramenta intermediária responsável por transformar dependencias de entidades em requisições para os serviços.
	- Além disso, o que a ferramenta promove?
		- Experimentação, adoçao e composição de APIs
		- Desenvolvimento continuo no cliente
		- Documentação de API’s em formatos machine-readable
		- Despreocupação em entender o fluxo de dados
		- Aumentar tempo de resposta diminuindo o tráfego e dados.
		- Avisos sobre problemas e melhorias na comunicação.
		- Uso de microserviços ao invés de serviços monoliticos

  Its been slowly creeping up on us, creating exciting new possibilities for our applications; APIs are changing the face of the Web.

  The web has essentially become a service oriented platform, where information and functionality is a available through an API; the Web succeeded where the enterprise largely failed.

  This success can be attributed to the fact that the web has been decentralized in its approach and has adopted less stringent technologies to become service oriented. Many early APIs were written using SOAP but now REST is the dominant force (though some are more REST than others).  The publication of REST APIs has been rapidly increasing.

  Some offer both SOAP and REST APIs, but this practice has been on the decline and REST is now preferred for most new APIs.

  [How REST replaced SOAP on the Web: What it means to you]
  [http://www.infoq.com/articles/rest-soap]

\chapter{Introdução}

% - O crescimento das “API’s”
% 	- Estatisticas sobre o mercado
% 		- numeros sobre crescimento
% 		- numeros sobre distribuiçao de estilos
% 		- numeros sobre projeções futuras
% 	- Por que ocorreu isso?
% 		- SOA is the way to go
% 		- “dev is shifting more and more to the client.”
% 		- companies want get faster into new platforms.
% 	- Qual problema isso está causando?
% 		- Clientes dependentes de tipos/estilos de serviço
% 	- Por que isso é ruim?
% 		- (Perf) Estilos de API podem estar sendo sub/super utilizadas devido a constante mudança fluxo de dados em que clientes vivem.
% 		- (Inovação) Investir em novos modelos/tipos/estilos requer tempo para analisar o impacto no cliente.
% 		- (Agile) APIs tendem a crescer verticalmente e adicionar features em serviços “monoliticos” podem atrasar o crescimento de uma organização.
% 	- Como resolver isso?
% 		- Impedir que o cliente tenha acesso direto as APIs
% 	- De que forma?
% 		- Criar uma ferramenta intermediária responsável por transformar dependencias de entidades em requisições para os serviços.
% 	- Além disso, o que a ferramenta promove?
% 		- Experimentação, adoçao e composição de APIs
% 		- Desenvolvimento continuo no cliente
% 		- Documentação de API’s em formatos machine-readable
% 		- Despreocupação em entender o fluxo de dados
% 		- Aumentar tempo de resposta diminuindo o tráfego e dados.
% 		- Avisos sobre problemas e melhorias na comunicação.
% 		- Uso de microserviços ao invés de serviços monoliticos

%   Its been slowly creeping up on us, creating exciting new possibilities for our applications; APIs are changing the face of the Web.

%   The web has essentially become a service oriented platform, where information and functionality is a available through an API; the Web succeeded where the enterprise largely failed.

%   This success can be attributed to the fact that the web has been decentralized in its approach and has adopted less stringent technologies to become service oriented. Many early APIs were written using SOAP but now REST is the dominant force (though some are more REST than others).  The publication of REST APIs has been rapidly increasing.

%   Some offer both SOAP and REST APIs, but this practice has been on the decline and REST is now preferred for most new APIs.

%   [How REST replaced SOAP on the Web: What it means to you]
%   [http://www.infoq.com/articles/rest-soap]

\section[Descrição do Problema]{Descrição do Problema}

O mercado de API's web transformou o modo de comunicação entre aplicações distribuídas para busca de informações e execução de operações na web. Não é de hoje que organizações tem se preocupado em disponibilizar API's de suas aplicações em forma de serviços para consumo próprio e/ou de terceiros. Empresas como Facebook e Netflix mostram que construir uma aplicação em forma de serviço e disponibilizar uma interface de acesso é essencial para entrar rápido no mercado de plataformas emergentes, oferecer uma melhor experiência para o usuário através do desenvolvimento de clientes nativos, além de agregar valor em seu modelo de negócio ao disponibilizar dados e operações para o uso de terceiros. (ART, 2016)

Segundo ProgrammableWeb, um dos motivos que contribuiu para o aumento do número de API's de serviços foi após a introdução em 2001 do estilo de arquitetura REST. Contudo, após 15 anos de sua introdução e diversos clientes escritos com base em seu protocolo de comunicação, REST tem-se mostrado uma solução ineficiente para lidar com o acesso de estrutura de dados por um grande número de clientes diversificado. Visto que sua interface de acesso é pré-determinada, após publicado, tornando-se difícil prever a demanda de dados e realizar mudanças no fluxo sem que haja versionamento. (ProgrammableWeb, 2016)

Por outro lado, ferramentas como GraphQL (Facebook) e Falcor (Netflix) surgem com o objetivo de resolver este problema mas encontram uma baixa adoção por não conseguirem minimizar a custosa transição que clientes desenvolvidos em REST precisam passar para fugir desta arquitetura.

\section[Objetivos]{Objetivos}

O objetivo deste trabalho é o desenvolvimento de uma ferramenta que ajude na prevenção do acoplamento entre clientes e API's causado pela escrita de código voltado ao acesso direto de estruturas de dados JSON. Além disso, através da análise de metadados, poder oferecer uma maneira genérica (usando GraphQL) para busca dados independente do estilo de arquitetura e fluxo de dados implementado pela API. \\

\textbf{Objetivos específicos} \\

\begin{itemize}
\item Prevenir a mudança de código em clientes devido à alterações na implementação do fluxo de dados por uma API.
\item Facilitar a transição de código em clientes ao atualizar para uma nova versão implementado por uma API.
\item Facilitar a transição de código em clientes ao atualizar para um novo estilo de arquitetura implementado por uma API.
\item Oferecer uma maneira de composição de estruturas de dados entre mais de um serviço e interface de aplicação.
\item Promover a documentação e descrição de metadados em API's. \\
\end{itemize}

\textbf{Limites da pesquisa} \\

\begin{itemize}
\item Foco exclusivo no formato JSON para representação de dados.
\item Foco exclusivo na linguagem GraphQL para consulta de dados.
\item Foco exclusivo no estilo de arquitetura REST para realizar testes.
\item Foco exclusivo na ferramenta JSON Hyper-Schema para descrição de API's REST.
\item Foco exclusivo em validar a ferramenta através de testes para prevenir a mudança de código em clientes devido à alterações na implementação do fluxo de dados por uma API.
\end{itemize}
\section[Metodologia]{Metodologia}

Problema surgiu ao tentar integrar GraphQL em um cliente que fazia requisições à uma API RESTful. Cliente estava completamente acoplado ao fluxo de dados, sendo que a estrutura de dados era a mesma. Como resolver isso?

Prox etapa foi a idealização do projeto. Estudar se era possível resolver esse problema e de que forma. Como criar uma ferramenta que fizesse essa intermediação. GraphQL se tornou uma dependência essencial para o sucesso da ferramenta.

Seguido, foi pensado na criação de uma interface para a ferramenta que fosse fácil de usar, rodasse em serviços GraphQL existentes e pudesse funcionar em API's REST que possuem descrição em linguagens/formatos populares e completos.

Após isso foi feito a implementação para Proof of Concept. Depois que validado, comecei a escrita da monografia e preparação de um ambiente de validação que pudesse enfatizar problemas reais na mudança fluxo de dados entre cliente e servidor. Coletar os dados, analisar e apresentar para comprovação da tese.
\section[Organização do Texto]{Organização do Texto}

O texto está organizado em 6 capítulos. O primeiro aborda os conceitos de fundamentos utilizados para entender o processo de desenvolvimento da ferramenta. Em seguida, em um capítulo a parte, será descrito a tecnologia GraphQL. Após os fundamentos, será descrito todo o processo de desenvolvimento da ferramenta, como a planejamento do projeto, implementação e validação. Após definir o ambiente de validação, será executado os testes de validação, coletado os dados e analisados no capítulo de resultados. Por fim, no último capítulo, será feito um comentário geral sobre o projeto, além de propor melhorias a serem feitas na ferramenta para trabalhos futuros.


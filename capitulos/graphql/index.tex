\chapter{GraphQL}

GraphQL é, além de um interpretador, uma linguagem de consulta de dados criado pelo Facebook para trabalhar com APIs de forma alternativa. Seu objetivo é fornecer uma descrição completa e compreensível de dados disponíveis em interfaces de aplicação. Assim, permitindo que clientes façam consultas de forma precisa em busca de dados que desejam trabalhar. \cite{GraphQL2016}

Por ser uma especificação recente, após sua publicação em 2015, GraphQL já apresenta implementações em diversas linguagens de programação e atualmente é utilizado em diversos contextos como na comunicação entre cliente-servidor, microserviços, navegação de árvores, gerador de consultas para banco de dados, entre outros.

É importante ressaltar que GraphQL não é uma linguagem de banco de dados, por mais que possa ser utilizado para esta finalidade. Ao invés, sua linguagem e interpretador trabalham com a ideia de mapeamento de campos e tipos de dados em aplicações, fornecendo uma interface unificada e amigável para desenvolvimento de produtos e de outras ferramentas, como a proposta neste trabalho. \cite{GraphQL2016}

Para gerar um esquema, antes é preciso que haja uma aplicação para que possa ser definido os tipos de estruturas, campos de acesso e funções de retorno de dados.

\begin{figure}[H]
  \centering
  \begin{minted}[frame=single,framesep=10pt,fontsize=\footnotesize]{text}
    var pessoa = {
      nome: "Mateus Maso"
    }

    class Query {
      pessoa() {
        return pessoa
      }
    }

    class Pessoa {
      nome(pessoa) {
        return pessoa.nome
      }
    }
  \end{minted}
  \caption{Exemplo de API em JavaScript (ES6)}
\end{figure}

\begin{figure}[H]
  \centering
  \begin{minted}[frame=single,framesep=10pt,fontsize=\footnotesize]{text}
    type Query {
      pessoa: User
    }

    type Pessoa {
      id: ID
      nome: String
    }
  \end{minted}
  \caption{Esquema GraphQL para API da Figura 11}
\end{figure}

Após a criação do esquema, já é possível enviar consultas GraphQL para que a ferramenta valide a semântica e sintaxe, transforme em execuções de API e retorne dados representados no formato JSON.  \cite{GraphQL2016}

\begin{figure}[H]
  \centering
  \begin{minted}[frame=single,framesep=10pt,fontsize=\footnotesize]{text}
    query {
      pessoa {
        nome
      }
    }
  \end{minted}
  \caption{Exemplo de consulta GraphQL para Figura 12}
\end{figure}

\begin{figure}[H]
  \centering
  \begin{minted}[frame=single,framesep=10pt,fontsize=\footnotesize]{text}
    {
      "pessoa": {
        "nome": "Mateus Maso"
      }
    }
  \end{minted}
  \caption{Resposta JSON esperada após execução pela Figura 13}
\end{figure}

A seguir, será feita uma abordagem sobre os elementos que compõem a linguagem, seu sistema de tipagem e como a analisar metadados de esquemas utilizando o processo de introspecção.

\subsubsection[Sem Estado]{Sem Estado}

Como HTTP é um protocolo sem conexão (onde não há nenhuma garantia sobre qual servidor será processado e quanto tempo irá levar) cada requisição deve conter todas as informações necessarias para que um servidor entenda o que um cliente está executando. Ou seja, para ser stateless, um servidor não pode guardar informações de estado do cliente, como sessões por exemplo. \cite{Fielding2000}

Esta restrição ajuda na viabilidade, confiabilidade e escalabilidade de sistemas distribuídos. Garantem que chamadas da API não estejam vinculadas a um determinado servidor. Contudo, com base no número da diversidade de clientes, ao manter um servidor sem estado é possível perder controle no tamanho de resposta, o que pode ser um fator crucial para aplicações que dependem disso. \cite{Wildermuth2015}

\subsubsection[Sem Estado]{Sem Estado}

Como HTTP é um protocolo sem conexão (onde não há nenhuma garantia sobre qual servidor será processado e quanto tempo irá levar) cada requisição deve conter todas as informações necessarias para que um servidor entenda o que um cliente está executando. Ou seja, para ser stateless, um servidor não pode guardar informações de estado do cliente, como sessões por exemplo. \cite{Fielding2000}

Esta restrição ajuda na viabilidade, confiabilidade e escalabilidade de sistemas distribuídos. Garantem que chamadas da API não estejam vinculadas a um determinado servidor. Contudo, com base no número da diversidade de clientes, ao manter um servidor sem estado é possível perder controle no tamanho de resposta, o que pode ser um fator crucial para aplicações que dependem disso. \cite{Wildermuth2015}

\section[Introspecção]{Introspecção}

Em GraphQL, o termo introspecção refere-se à uma consulta especial usada na busca por metadados de esquemas. Uma das vantagens desta funcionalidade está na possibilidade de ferramentas usá-la para gerar código, documentação, prever mudanças, alertar de campos deprecados, entre outras aplicações. Isso torna GraphQL uma ótima opção como estilo de arquiteturas em sistemas distribuídos, pois apresenta de forma "nativa" uma poderosa linguagem de descrição de serviços.

\begin{figure}[H]
  \centering
  \begin{minted}[frame=single,framesep=10pt,fontsize=\small]{javascript}
    query introspeccao {
      __schema {
        queryType { name }
        mutationType { name }
        types {
          kind
          name
          description
          fields {
            name
            description
          }
        }
      }
    }
  \end{minted}
  \caption{Introspecção}
\end{figure}


\subsection[JSON Schema]{JSON Schema}

JSON Schema é uma linguagem de definição projetada para descrever estruturas de dados JSON por meio de esquemas. Foi proposta em 2009 por Kris Zyp com objetivo de fornecer um contrato para que aplicações soubessem como trabalhar e interagir com estruturas de dados. Por meio deste, é possível prever representações e assim realizar operações de validação, documentação, navegação hyperlink e controle de iteração.

Por ser uma linguagem de simples uso, para modelar um esquema basta construir um objeto JSON utilizando um subconjunto válido de chaves especias descritas pela linguagem. No entanto, funcionalidades como descrição de estruturas multimídia\footnote{
  Imagens, videos, audio digital.
} e a navegação de dados são apenas disponibilizadas no formato JSON Hyper-Schema, uma variação da linguagem de especificação. \cite{Jackson2016}

\begin{figure}[H]
  \centering
  \begin{minted}[frame=single,framesep=10pt]{javascript}
    {
      "\$schema": "http://json-schema.org/draft-04/schema#",
      "title": "Pessoa",
      "description": "Uma pessoa",
      "type": "object",
      "required": ["nome", "aniversario"],
      "properties": {
        "nome": {
          "type": "string"
        },
        "aniversario": {
          "type": "string"
        },
        "cidade": {
          "type": "string"
        }
      }
    }
  \end{minted}
  \caption{JSON Schema para Figura 3}
\end{figure}

\begin{figure}[H]
  \centering
  \begin{minted}[frame=single,framesep=10pt]{javascript}
    {
      "\$schema": "http://json-schema.org/draft-04/hyper-schema#",
      ...
      "properties": {
        ...
        "foto": {
          "media": {
            "binaryEncoding": "base64",
            "type": "image/png"
          }
        }
      },
      "links": [
        {
          "rel": "foto",
          "href": "/{id}.png",
          "mediaType": "image/png"
        }
      ]
    }
  \end{minted}
  \caption{JSON Hyper-Schema para Figura 3}
\end{figure}

Ao exemplo das figuras 5 e 6, ambos os esquemas asseguram que, dado uma estrutura JSON, para que esta seja reconhecido como uma entidade “pessoa” deve conter as propriedades "nome" e "aniversario" com valores do tipo "string". Já na figura 6, além das estruturas definidas pela figura 5, é descrito uma nova propriedade do tipo multimídia e como navegar em busca desta informação.

Em casos onde a complexidade de um esquema começa a crescer, é comum a definição de subesquemas através da chave “definitions”. Desta forma, podem ser referenciados pela chave "\$ref" permitindo o reuso de estruturas dentro de um esquema. Vale lembrar que a chave “definitions” é apenas um mecanismo simples e util para definir esquemas em um lugar comum, entretanto, não sugerem que estas propriedades sejam validadas em um objeto ao menos que referenciadas em outras estruturas do esquema. \cite{Leach2014}

\begin{figure}[H]
  \centering
  \begin{minted}[frame=single,framesep=10pt]{javascript}
    {
      "\$schema": "http://json-schema.org/draft-04/hyper-schema#",
      ...
      "definitions": {
        "cidade": {
          "type": "string",
          "properties": {
            "nome": { "type": "string" },
            "estado": { "type": "string" },
            "pais": { "type": "string" }
          }
        }
      },
      "properties": {
        ...
        "cidade": {
          "\$ref": "#/definitions/cidade"
        }
      },
      "links": [
        ...
        {
          "rel": "cidade",
          "href": "/{id}/cidade",
          "targetSchema": {
            "\$ref": "#/definitions/cidade"
          }
        }
      ]
    }
  \end{minted}
  \caption{JSON Hyper-Schema para Figura 4 usando \$ref}
\end{figure}

Como boa prática, é recomendado (mas não necessário) o uso da chave especial “\$schema” para determinar quando uma estrutura JSON está sendo representada em forma de esquema. A seguir, será descrito algumas das chaves especiais usadas para descrever objetos em esquemas. \cite{Droettboom2015}

\begin{table}[ht!]
  \centering
  \resizebox{\columnwidth}{!}{
    \begin{tabular}{|c|c|}
      \hline
      Chave & Descrição \\
      \hline
      \$schema & Identificador de versão \\
      \hline
      type & Tipo de dado \\
      \hline
      title & Nome da estrutura \\
      \hline
      description & Propósito da estrutura \\
      \hline
      required & Lista de propriedades com presença obrigatória \\
      \hline
      properties & Propriedades usadas para validar uma estrutura \\
      \hline
      definitions & Propriedades (subesquemas) para referência \\
      \hline
      ... & ... \\
      \hline
      links & Lista de Link Description Objects (LDO) \\
      \hline
    \end{tabular}
  }
  \caption{Subconjunto de chaves especiais JSON Schema}
\end{table}

De certa forma, JSON Schema continua sendo uma das únicas tentativas sérias de propor uma linguagem de definição em esquemas para estruturas JSON e, lentamente, está sendo estabelecida como padrão de uso para especificação JSON. Contudo, sua definição ainda está longe de ser um padrão universal, mas já há um número crescente de aplicações que suportam JSON Schema e uma quantidade significativa de ferramentas que permitem a validação de JSON Schemas. \cite{PezoaEtAl2016}

Vale lembrar também que, segundo Leach, com o recente surgimento de grandes formatos de descrição de APIs ao longo dos últimos anos como Swagger, Blueprint e RAML. JSON Schema tem-se tornado uma ótima opção de tecnologia para descrever API's uma vez que utiliza da popularidade do formato JSON uma forma simples de descrever e disponibilizar esquemas pela web. Além disso, são extremamente nesses cenários, onde pode-se usar para evitar o recebimento de chamadas de APIs mal formadas que podem afetar o motor interno da aplicação. \cite{Leach2014} \cite{PezoaEtAl2016}

Nos próximos capítulos vamos entrar mais em detalhe sobre tipos de serviços (API web) e de que forma é possível descrever APIs para o leitura por máquinas.

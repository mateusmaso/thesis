\subsection[Sistema de Tipagem]{Sistema de Tipagem}

Para descrever estruturas de dados de API's utilizando esquema GraphQL, é preciso fazer o uso de seu sistema de tipagem. Através da abstração de entidades de um serviço, representa-se um conjunto finito de tipos, relacionamentos e diretivas para ser acessado por clientes através de documentos GraphQL.

Tipos podem ser classificados como abstrato, folha, de entrada, de saída e para composição.  Folha é blabla. Composição é blabla, elementar para representar entidades e relacionamentos. Abstrato é blabla. Entrar e saída são blabla.

\begin{table}[H]
  \centering
  \begin{tabular}{|c|c|c|c|c|}
    \hline
    Tipo & Categoria \\
    \hline
    Enum & Folha \\
    \hline
    Int & Folha \\
    \hline
    Float & Folha \\
    \hline
    String & Folha \\
    \hline
    Boolean & Folha \\
    \hline
    ID & Folha \\
    \hline
    Object & Composição \\
    \hline
    Union & Abstrato \\
    \hline
    Interface & Abstrato \\
    \hline    
    Non-Null & Entrada e Saída \\
    \hline
  	List & Entrada e Saída \\
    \hline
  \end{tabular}
  \caption{Classificação de tipos GraphQL}
\end{table}

The most basic type is a Scalar. A scalar represents a primitive value, like a string or an integer. Oftentimes, the possible responses for a scalar field are enumerable. Scalars and Enums form the leaves in response trees; the intermediate levels are Object types, which define a set of fields. GraphQL supports two abstract types: interfaces and unions. An Interface defines a list of fields; Object types that implement that interface are guaranteed to implement those fields. All of the types so far are assumed to be both nullable and singular. The type system might want to define that it returns a list of other types; the List type is provided for this reason, and wraps another type. Similarly, the Non-Null type wraps another type, and denotes that the result will never be null. These two types are referred to as “wrapping types”; non-wrapping types are referred to as “base types”. A wrapping type has an underlying “base type”, found by continually unwrapping the type until a base type is found.

Vale lembrar que, para serem válidos em um esquema, precisam ser incluídos em um tipo especial chamado "root", este é usado como principal nó de entrada para cada operação (consulta e mutação).

\begin{figure}[H]
  \centering
  \begin{minted}[frame=single,framesep=10pt,fontsize=\small]{javascript}
    schema => interfaces (=> |individuo|), types  (=> |pessoa|)
  \end{minted}
  \caption{Demonstração um esquema GraphQL}
\end{figure}

\begin{figure}[H]
  \centering
  \begin{minted}[frame=single,framesep=10pt,fontsize=\small]{javascript}
    interface Individuo {
      nome: String
    }

    type Pessoa implements Individuo {
      ano: Int
      foto: Foto
      amigos: [Pessoa]
    }

    type Foto {
      altura: Int
      largura: Int
    }

    union Resultado = Foto | Pessoa

    type Pesquisa {
      resultado: Resultado
    }
  \end{minted}
  \caption{Exemplo de representação do esquema da figura 17 em GraphQL}
\end{figure}

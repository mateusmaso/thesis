\section{GraphQL}

GraphQL é uma linguagem de consulta de dados e interpretador de consultas para APIs. Foi desenvolvida pelo Facebook em 2012 mas sua especificação apenas publicada em 2015. Tem com objetivo fornecer uma descrição completa e compreensível de dados disponíveis em interfaces de aplicação. Além disso, permite que clientes façam consultas para busca exata de dados que desejam trabalhar. \cite{GraphQL2016}

Importante ressaltar que GraphQL não é uma linguagem para banco de dados, por mais que possa ser utilizado para esta finalidade. Ao invés, sua linguagem e interpretador trabalham com a idéia de mapeamento de campos e tipos de dados em aplicações, fornecendo uma interface unificada e amigável para desenvolvimento de produtos, além de uma plataforma poderosa para a construção de ferramentas. \cite{GraphQL2016}

Por ser uma especificação, GraphQL possui implementações em diversas linguagens de programação e atualmente é usado em diversos contextos, como comunicação entre cliente-servidor, microserviços, simplificação de APIs, navegação de árvores, gerador de consultas para banco de dados, entre outros.

Para gerar um esquema GraphQL, antes é preciso definir os tipos de estruturas, seus campos de acesso e funções para mapear e retornar dados reais em código.

\begin{figure}[H]
  \centering
  \inputminted[frame=single,framesep=10pt]{javascript}{anexos/pessoa.js}
  \caption{API de um código em JavaScript}
\end{figure}

\begin{figure}[H]
  \centering
  \inputminted[frame=single,framesep=10pt]{javascript}{anexos/pessoa-graphql-schema.graphql}
  \caption{Esquema GraphQL para API em JavaScript}
\end{figure}

Uma vez que o esquema GraphQL foi gerado, pode ser enviado consultas GraphQL para construção de estruturas de dados. Primeiramente, é feito a validação da consulta para garantir que esta só se refere aos tipos e campos definidos, em seguida, executa as funções fornecidas para produzir um resultado. \cite{GraphQL2016}

\begin{figure}[H]
  \centering
  \inputminted[frame=single,framesep=10pt]{javascript}{anexos/pessoa-graphql-query.graphql}
  \caption{Query GraphQL para esquema}
\end{figure}

\begin{figure}[H]
  \centering
  \inputminted[frame=single,framesep=10pt]{javascript}{anexos/pessoa-graphql-query-response.json}
  \caption{Resposta JSON da Query GraphQL}
\end{figure}

\subsection[Linguagem de Consulta]{Linguagem de Consulta}

Clientes que buscam realizar consultas de dados em serviços GraphQL precisam antes entender seu formato de requisição, também chamado de documento. Um documento contém operações como consultas e mutações. Além disso, é possível especificar fragmentos, uma unidade comum de composição para reuso de consultas. \cite{GraphQL2016}

\textbf{Sintaxe}

A GraphQL query document describes a complete file or request string received by a GraphQL service. A document contains multiple definitions of Operations and Fragments. GraphQL query documents are only executable by a server if they contain an operation. If a document contains only one operation, that operation may be unnamed or represented in the shorthand form, which omits both the query keyword and operation name. Otherwise, if a GraphQL query document contains multiple operations, each operation must be named.

There are two types of operations that GraphQL models: query a read-only fetch. mutation, a write followed by a fetch. Each operation is represented by an optional operation name and a selection set. If a document contains only one query operation, and that query defines no variables and contains no directives, that operation may be represented in a short-hand form which omits the query keyword and query name. An operation selects the set of information it needs, and will receive exactly that information and nothing more, avoiding over-fetching and under-fetching data. In this query, the id, firstName, and lastName fields form a selection set. Selection sets may also contain fragment references.

A selection set is primarily composed of fields. A field describes one discrete piece of information available to request within a selection set. Some fields describe complex data or relationships to other data. In order to further explore this data, a field may itself contain a selection set, allowing for deeply nested requests. All GraphQL operations must specify their selections down to fields which return scalar values to ensure an unambiguously shaped response. Fields are conceptually functions which return values, and occasionally accept arguments which alter their behavior. These arguments often map directly to function arguments within a GraphQL server’s implementation.

\begin{figure}[H]
  \centering
  \begin{minted}[frame=single,fontsize=\small]{javascript}
    {
      pessoa(id: 4) {
        id
        nome
        sobrenome
        nascimento: aniversario {
          mes
          dia
        }
        amigos(limite: 10) {
          nome
        }
      }
    }
  \end{minted}
  \caption{Sintaxe}
\end{figure}

\textbf{Fragmentos}

Fragments are the primary unit of composition in GraphQL. Fragments allow for the reuse of common repeated selections of fields, reducing duplicated text in the document. Inline Fragments can be used directly within a selection to condition upon a type condition when querying against an interface or union.

Fragments must specify the type they apply to. In this example, friendFields can be used in the context of querying a User. Fragments cannot be specified on any input value (scalar, enumeration, or input object).Fragments can be specified on object types, interfaces, and unions.

Fragments can be defined inline within a selection set. This is done to conditionally include fields based on their runtime type. This feature of standard fragment inclusion was demonstrated in the query FragmentTyping example. We could accomplish the same thing using inline fragments.

\begin{figure}[H]
  \centering
  \begin{minted}[frame=single,fontsize=\small]{javascript}
    {
      pessoa(id: 4) {
        ...identidade
        ... on User {
          friends {
            name
          }
        }
      }
    }

    fragment identidade on Pessoa {
      id
      name
    }
  \end{minted}
  \caption{Fragmentos}
\end{figure}

\subsection[Sistema de Tipagem]{Sistema de Tipagem}

Para descrever estruturas de dados de API's utilizando esquema GraphQL, é preciso fazer o uso de seu sistema de tipagem. Através da abstração de entidades de um serviço, representa-se um conjunto finito de tipos, relacionamentos e diretivas para ser acessado por clientes através de documentos GraphQL.

Tipos podem ser classificados como abstrato, folha, de entrada, de saída e para composição.  Folha é blabla. Composição é blabla, elementar para representar entidades e relacionamentos. Abstrato é blabla. Entrar e saída são blabla.

\begin{table}[H]
  \centering
  \begin{tabular}{|c|c|c|c|c|}
    \hline
    Tipo & Categoria \\
    \hline
    Enum & Folha \\
    \hline
    Int & Folha \\
    \hline
    Float & Folha \\
    \hline
    String & Folha \\
    \hline
    Boolean & Folha \\
    \hline
    ID & Folha \\
    \hline
    Object & Composição \\
    \hline
    Union & Abstrato \\
    \hline
    Interface & Abstrato \\
    \hline    
    Non-Null & Entrada e Saída \\
    \hline
  	List & Entrada e Saída \\
    \hline
  \end{tabular}
  \caption{Classificação de tipos GraphQL}
\end{table}

The most basic type is a Scalar. A scalar represents a primitive value, like a string or an integer. Oftentimes, the possible responses for a scalar field are enumerable. Scalars and Enums form the leaves in response trees; the intermediate levels are Object types, which define a set of fields. GraphQL supports two abstract types: interfaces and unions. An Interface defines a list of fields; Object types that implement that interface are guaranteed to implement those fields. All of the types so far are assumed to be both nullable and singular. The type system might want to define that it returns a list of other types; the List type is provided for this reason, and wraps another type. Similarly, the Non-Null type wraps another type, and denotes that the result will never be null. These two types are referred to as “wrapping types”; non-wrapping types are referred to as “base types”. A wrapping type has an underlying “base type”, found by continually unwrapping the type until a base type is found.

Vale lembrar que, para serem válidos em um esquema, precisam ser incluídos em um tipo especial chamado "root", este é usado como principal nó de entrada para cada operação (consulta e mutação).

\begin{figure}[H]
  \centering
  \begin{minted}[frame=single,framesep=10pt,fontsize=\small]{javascript}
    schema => interfaces (=> |individuo|), types  (=> |pessoa|)
  \end{minted}
  \caption{Demonstração um esquema GraphQL}
\end{figure}

\begin{figure}[H]
  \centering
  \begin{minted}[frame=single,framesep=10pt,fontsize=\small]{javascript}
    interface Individuo {
      nome: String
    }

    type Pessoa implements Individuo {
      ano: Int
      foto: Foto
      amigos: [Pessoa]
    }

    type Foto {
      altura: Int
      largura: Int
    }

    union Resultado = Foto | Pessoa

    type Pesquisa {
      resultado: Resultado
    }
  \end{minted}
  \caption{Exemplo de representação do esquema da figura 17 em GraphQL}
\end{figure}

\subsection[Instrospecção]{Instrospecção}

A GraphQL server supports introspection over its schema. This schema is queried using GraphQL itself, creating a powerful platform for tool-building. Take an example query for a trivial app. In this case there is a User type with three fields: id, name, and birthday. For example, given a server with the following type definition:

\begin{figure}[H]
  \centering
  \begin{minted}[frame=single,framesep=10pt,fontsize=\small]{javascript}
    query introspeccao {
      __schema {
        queryType { name }
        mutationType { name }
        types {
          kind
          name
          description
          fields {
            name
            description
          }
        }
      }
    }
  \end{minted}
  \caption{Introspecção}
\end{figure}


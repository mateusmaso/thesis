\section{GraphQL}

GraphQL é uma linguagem de consulta de dados e interpretador para APIs. Foi desenvolvida pela Facebook em 2012 mas sua especificação apenas publicada em 2015. Tem com objetivo fornecer uma descrição completa e compreensível dos dados disponíveis em APIs. Além disso, dá aos clientes o poder de trabalhar apenas com as estrutura de dados que precisam, torna mais fácil evoluir APIs ao longo do tempo e permite o desenvolvimento de ferramentas em cima de sua linguagem. \cite{GraphQL2016}

Importante ressaltar que GraphQL não está preso a algum banco de dados especifico ou mecanismo de armazenamento. Ao invés, é utilizado para consultar código já existente ou estruturas de dados. Para criar um esquema, é preciso definir os tipos de estruturas, seus campos de acesso e funções para mapear e retornar dados reais em código. \cite{GraphQL2016}

Por ser uma especificação, GraphQL possui implementações em diversas linguagens de programação e atualmente é usado em diversos contextos, sendo eles para comunicação entre cliente-servidor, microserviços, simplificação de APIs, navegação de árvores, gerador de consultas para banco de dados, entre outros.

\begin{figure}[h]
  \centering
  \inputminted[frame=single,framesep=10pt]{javascript}{anexos/graphql-schema.gql}
  \caption{Esquema GraphQL}
\end{figure}

\begin{figure}[h]
  \centering
  \inputminted[frame=single,framesep=10pt]{javascript}{anexos/graphql-code.js}
  \caption{API de um código em JavaScript}
\end{figure}

Uma vez que um serviço GraphQL está sendo executado (normalmente a uma URL em um serviço web), pode ser enviado consultas GraphQL para validar e executar. Uma consulta recebida é primeiro verificado para garantir que ele só se refere aos tipos e campos definidos, em seguida, executa as funções fornecidas para produzir um resultado. \cite{GraphQL2016}

\begin{figure}[h]
  \centering
  \inputminted[frame=single,framesep=10pt]{javascript}{anexos/graphql-query.gql}
  \caption{Query GraphQL para esquema}
\end{figure}

\begin{figure}[h]
  \centering
  \inputminted[frame=single,framesep=10pt]{javascript}{anexos/graphql-query-response.json}
  \caption{Resposta JSON da Query GraphQL}
\end{figure}

\subsection[Linguagem de Consulta]{Linguagem de Consulta}

Clientes que buscam realizar consultas de dados em serviços GraphQL precisam antes entender seu formato de requisição, também chamado de documento. Um documento contém operações como consultas e mutações. Além disso, é possível especificar fragmentos, uma unidade comum de composição para reuso de consultas. \cite{GraphQL2016}

\subsubsection[Sintaxe]{Sintaxe}

A GraphQL query document describes a complete file or request string received by a GraphQL service. A document contains multiple definitions of Operations and Fragments. GraphQL query documents are only executable by a server if they contain an operation. If a document contains only one operation, that operation may be unnamed or represented in the shorthand form, which omits both the query keyword and operation name. Otherwise, if a GraphQL query document contains multiple operations, each operation must be named.

There are two types of operations that GraphQL models: query – a read‐only fetch. mutation – a write followed by a fetch. Each operation is represented by an optional operation name and a selection set. If a document contains only one query operation, and that query defines no variables and contains no directives, that operation may be represented in a short‐hand form which omits the query keyword and query name. An operation selects the set of information it needs, and will receive exactly that information and nothing more, avoiding over‐fetching and under‐fetching data. In this query, the id, firstName, and lastName fields form a selection set. Selection sets may also contain fragment references.

A selection set is primarily composed of fields. A field describes one discrete piece of information available to request within a selection set. Some fields describe complex data or relationships to other data. In order to further explore this data, a field may itself contain a selection set, allowing for deeply nested requests. All GraphQL operations must specify their selections down to fields which return scalar values to ensure an unambiguously shaped response. Fields are conceptually functions which return values, and occasionally accept arguments which alter their behavior. These arguments often map directly to function arguments within a GraphQL server’s implementation.

\begin{figure}[H]
  \centering
  \inputminted[frame=single,framesep=10pt]{javascript}{anexos/graphql-syntax.txt}
  \caption{Sintaxe}
\end{figure}

\subsubsection[Fragmentos]{Fragmentos}

Fragments are the primary unit of composition in GraphQL. Fragments allow for the reuse of common repeated selections of fields, reducing duplicated text in the document. Inline Fragments can be used directly within a selection to condition upon a type condition when querying against an interface or union.

Fragments must specify the type they apply to. In this example, friendFields can be used in the context of querying a User. Fragments cannot be specified on any input value (scalar, enumeration, or input object).Fragments can be specified on object types, interfaces, and unions.

Fragments can be defined inline within a selection set. This is done to conditionally include fields based on their runtime type. This feature of standard fragment inclusion was demonstrated in the query FragmentTyping example. We could accomplish the same thing using inline fragments.

\begin{figure}[H]
  \centering
  \begin{minted}[frame=single,framesep=10pt]{javascript}
    {
      pessoa(id: 4) {
        ...identidade
        ... on User {
          friends {
            name
          }
        }
      }
    }

    fragment identidade on Pessoa {
      id
      name
    }
  \end{minted}
  \caption{Fragmentos}
\end{figure}


\subsection[Sistema de Tipagem]{Sistema de Tipagem}

The GraphQL Type system describes the capabilities of a GraphQL server and is used to determine if a query is valid. The type system also describes the input types of query variables to determine if values provided at runtime are valid. A GraphQL server’s capabilities are referred to as that server’s “schema”. A schema is defined in terms of the types and directives it supports.

A given GraphQL schema must itself be internally valid. A GraphQL schema is represented by a root type for each kind of operation: query and mutation; this determines the place in the type system where those operations begin. All types within a GraphQL schema must have unique names. No two provided types may have the same name. No provided type may have a name which conflicts with any built in types (including Scalar and Introspection types).

-------

(tipos folha: enumerados, escalares)
(tipos de entrada e saida) => lista, nao nulo, objetos de entrada
(tipos de composicao) => objetos
(tipos de abstracao) => interfaces e unicoes

The most basic type is a Scalar. A scalar represents a primitive value, like a string or an integer. Oftentimes, the possible responses for a scalar field are enumerable. GraphQL offers an Enum type in those cases, where the type specifies the space of valid responses. Scalars and Enums form the leaves in response trees; the intermediate levels are Object types, which define a set of fields, where each field is another type in the system, allowing the definition of arbitrary type hierarchies.

GraphQL supports two abstract types: interfaces and unions. An Interface defines a list of fields; Object types that implement that interface are guaranteed to implement those fields. Whenever the type system claims it will return an interface, it will return a valid implementing type. A Union defines a list of possible types; similar to interfaces, whenever the type system claims a union will be returned, one of the possible types will be returned.

All of the types so far are assumed to be both nullable and singular: e.g. a scalar string returns either null or a singular string. The type system might want to define that it returns a list of other types; the List type is provided for this reason, and wraps another type. Similarly, the Non-Null type wraps another type, and denotes that the result will never be null. These two types are referred to as “wrapping types”; non‐wrapping types are referred to as “base types”. A wrapping type has an underlying “base type”, found by continually unwrapping the type until a base type is found.

\begin{figure}[H]
  \centering
  \inputminted[frame=single,framesep=10pt]{javascript}{anexos/graphql-type-system.graphql}
  \caption{Sistema de Tipagem}
\end{figure}

\subsection[Instrospecção]{Instrospecção}

A GraphQL server supports introspection over its schema. This schema is queried using GraphQL itself, creating a powerful platform for tool-building. Take an example query for a trivial app. In this case there is a User type with three fields: id, name, and birthday. For example, given a server with the following type definition:

\begin{figure}[H]
  \centering
  \begin{minted}[frame=single,framesep=10pt,fontsize=\small]{javascript}
    query introspeccao {
      __schema {
        queryType { name }
        mutationType { name }
        types {
          kind
          name
          description
          fields {
            name
            description
          }
        }
      }
    }
  \end{minted}
  \caption{Introspecção}
\end{figure}


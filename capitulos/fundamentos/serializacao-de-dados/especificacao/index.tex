\subsection[Especificação]{Especificação}

Um formato pode ter sua especificação classificada como padronizada ou não padronizada. Uma especificação padronizada é regida por requisitos que auxiliam na reprodutibilidade do processo em outras linguagens para maximização da compatibilidade e minimização de erros. Ao contrário, dada uma linguagem de programação, não é garantido que sua implementação esteja seguindo os padrões e poderá ser considerado como não padronizada. \cite{McDermid1991}

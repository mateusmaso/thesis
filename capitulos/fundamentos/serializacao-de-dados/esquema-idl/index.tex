\subsection[Esquema/IDL]{Esquema/IDL}

Com objetivo de entender o que está sendo representado, alguns formatos disponibilizam na sua especificação maneiras de descrever metadados. Esta categoria é importante principalmente para que máquinas consigam inferir quais estruturas estão sendo lidas e, assim, tomar decisões de forma autônoma.

Um formato descritivo normalmente disponibiliza estruturas como esquemas IDL\footnote{
  Linguagem de descrição utilizada para descrever a interface dos componentes de um software.
} para descrição da própria representação. À medida que estas descrições são incorporadas dentro da mesma representação, é possível classificar estes formatos como sendo auto-descritivos. \cite{Rentachintala2014}

\subsubsection[Cliente-Servidor]{Cliente-Servidor}

É a primeira restrição para uma aplicação REST.  Neste modelo, não existe conexão entre cliente e servidor, mas sim a espera do servidor por pedidos de clientes, executando solicitações e devolvendo uma resposta. Seu objetivo é separar arquitetura e responsabilidades dos ambientes cliente e servidor. Assim, o cliente (consumidor do serviço) não se preocupa com tarefas como a comunicação de banco de dados, gerenciamento de cache, etc. O inverso também é verdadeiro, onde o servidor (prestador de serviços) não está preocupado com as tarefas do cliente como interface, experiência do usuário etc. Assim, permitindo a evolução independente das duas arquiteturas desde que a interface de comunicação entre os dois não seja alterada. \cite{Fielding2000}

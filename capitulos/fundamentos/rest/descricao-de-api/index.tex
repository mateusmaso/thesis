\subsection[Descrição de API]{Descrição de API}

% APIs REST tornaram-se padrão de interface de acesso à serviços de empresas. sobre a api web para permitir que usuarios descubram e entam o serviço tornou-se uma pratica comum

% ainda não atingiu um grau de maturidade para não normas amplamente aceitas. Apenas 

Para empresas de software que disponibilizam APIs REST como forma de produto, a capacidade de oferecer uma completa descrição e facilidade de uso tornou-se um fator critico para o sucesso da empresa. Diferente do processo de implementação, a prática de descrição de APIs ainda continua sendo feita em sua maior parte manualmente através de websites em linguagem natural. \cite{LuckyEtAl2016}

 permite que apenas pessoas entendam, isso se for bem projetada 
 
Apesar de ter se mostrado uma solução de fácil acesso em clientes web, mobile apps e IoTs\footnote{
  Internet of Things
}. REST não propõem nenhuma forma de documentação além do uso de HATEOS para descrição e navegação de dados em respostas hypertexto. Para Knupp, a idéia de HATEOS é questionável pois dificulta a legibilidade, cria complexidade e aumenta o tamanho das respostas. Ao invés, sugere o uso de ferramentas para documentação de APIs. \cite{Knupp2016}

Com o fracasso de formatos tradicionais para descrição de web services como WADL, a adoção duvidosa de formatos hypermedia de resposta HATEOS como HAL e a demanda cada vez mais alta por boas especificações em formato legivel por humanos e maquinas. Nos últimos anos foram introduzidas diversas ferramentas e formatos de descrição para descrever Web APIs de REST, tanto em formatos legíveis para humanos como para máquinas. \cite{LuckyEtAl2016}

A seguir, comparações feitas por Sandoval entre as 3 linguagens mais usadas para especificação de APIs: \cite{Sandoval2015}

\begin{description}[leftmargin=8em,style=nextline]
  \item[\textbf{OpenAPI} (Swagger)] Pros: Amplamente adotada, grande comunidade, suporte pra diversas linguagens. \\ Cons: Carece de especificações de metadados mais avançadas.
  \item[\textbf{RAML}] Pros: Suporte a especificação avançada, adoção significativa, formato legível, bom suporte da indústria. \\ Cons: Falta de ferramentas de auxílio, não comprovada a longo prazo.
  \item[\textbf{API Blueprint}] Pros: Fácil de entender, simples de escrever \\ Cons: Pouca adoção, Carece de especificações de metadados mais avançadas, instalação complexa.
\end{description}

Recentemente, empresas como Heroku tem se dedicado ao uso do JSON Schema como formato de descrição de APIs, por ser uma tecnologia limpa e ainda pouco aplicada especificamente para a construção de grandes API \cite{Leach2014}. Já Lynn propõe um método de modelagem de REST API em serviços IoT utilizando JSON Hyper-Schema visto nos capítulos anteriores. Com suporte a descrição de entrada e saída de dados em toda a interface, junto com descrições URIs, relações e métodos que se aplicam aos links. Além disso, o formato suporta HATEOS e serve como entrada de documentação e ferramenta para geração de código. \cite{LynnEtAl2016}

\begin{resumo}
  A fim de manter a eficiência da comunicação cliente-servidor sem a necessidade do versionamento de APIs Web, serviços têm encontrado dificuldades em realizar mudanças na especificação de sua interface de acesso devido ao acoplamento causado por clientes na implementação em seu código de busca. No intuito de desenvolver uma solução para o problema encontrado, este trabalho realiza um estudo sobre o uso da linguagem GraphQL e formatos de descrição de API para propor um modelo de comunicação cliente-servidor através da automação na execução de consultas de dados. Como resultado, é desenvolvida uma ferramenta prevista pelo modelo proposto para validar sua aplicabilidade e direcionar desenvolvedores de clientes à implementação de um código de busca independente de especificação de API. \\ \\
  \textbf{Palavras-chaves}: GraphQL. REST. JSON Hyper-Schema.
\end{resumo}

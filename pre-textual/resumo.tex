\begin{resumo}
  Em busca de acompanhar as crescentes transformações no fluxo de dados por clientes em APIs Web, serviços têm aplicado esforços para adaptar suas interfaces de acesso a fim de manter uma comunicação eficiente. Contudo, há dificuldades em gerenciar essas mudanças pois, dependendo do código de busca implementado pelos clientes, existe a possibilidade de comprometer a comunicação cliente-servidor. No intuito de desenvolver soluções para o problema de acoplamento, este projeto realiza um estudo sobre o uso da linguagem GraphQL e formatos de descrição de API. Um novo modelo de comunicação cliente-servidor é proposto para direcionar desenvolvedores de clientes à implementação de códigos de busca independente de especificação de API. Através da automação de consultas, o modelo visa ser tolerante às mudanças na especificação, otimizar requisições e facilitar a composição de serviços em busca de dados. Como resultado, é desenvolvida uma ferramenta com base no modelo proposto para validar sua aplicabilidade. \\ \\
  \textbf{Palavras-chaves}: GraphQL. REST. JSON Hyper-Schema.
\end{resumo}
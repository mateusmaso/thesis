\begin{resumo}
  Em busca de acompanhar as crescentes transformações por clientes na execução de consultas de dados sobre APIs Web, serviços têm aplicado esforços em atualizar especificações de suas APIs baseadas na demanda de acesso. No entanto, a fim de manter a eficiência da comunicação cliente-servidor sem criar versionamento, ainda há dificuldades dos serviços em realizar mudanças na API devido ao acoplamento causado pela implementação do código de busca de clientes na especificação de acesso. No intuito de desenvolver uma solução para o problema de acoplamento, este trabalho realiza um estudo sobre o uso da linguagem GraphQL e formatos de descrição de API para propor um modelo de comunicação cliente-servidor através da automação na execução de consultas de dados. Dessa forma, o modelo visa direcionar desenvolvedores de clientes à implementação de códigos de busca independente de especificação de API. Como resultado, é desenvolvida uma ferramenta prevista pelo modelo para validar sua aplicabilidade. \\ \\
  \textbf{Palavras-chaves}: GraphQL. REST. JSON Hyper-Schema.
\end{resumo}
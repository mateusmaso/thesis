\textoResumo{  
  Para acomodar a rápida transformação na demanda de dados por clientes de aplicações distribuídas, tem se discutido cada vez mais novas e eficientes formas de disponibilizar estruturas de dados em serviços para consumo de clientes. Contudo, as atuais formas de comunicação cliente-servidor tem deixado o lado do cliente dependente da implementação do fluxo de dados e estilo de arquitetura oferecida pela API do servidor. Isso porque clientes precisam se preocupar em escrever código voltado ao acesso direto de estruturas de dados, levando a um acoplamento de API muitas vezes indesejado. Para evitar isso, este trabalho realiza um estudo sobre o uso da linguagem GraphQL como solução no desenvolvimento de uma ferramenta para que clientes possam buscar e compor estruturas de dados JSON através de serviços sem a necessidade de entender interfaces de aplicação. Permitindo que equipes de desenvolvimento possam constantemente questionar, experimentar e realizar mudanças no fluxo de dados de API's com o objetivo de diminuir o tempo de resposta e tamanho de dados sem ter impacto no código dos clientes.
}
\palavrasChave{
  GraphQL,
  REST,
  JSON Schema.
}

\paginaresumo
\textoResumo{  
  Para acomodar a rápida transformação na demanda de dados por clientes de aplicações distribuídas, recentemente, tem se discutido novas e eficientes maneiras de se disponibilizar estruturas de dados para consumo de clientes diversificados. Contudo, as atuais formas de comunicação cliente-servidor tem causado uma dependência na implementação do fluxo de dados e estilo de arquitetura oferecida pela API. Isso porque, para o acesso de estruturas de dados, clientes precisam se preocupar em escrever código voltado à especificação atual da API, levando à um acoplamento muitas vezes indesejado. Para evitar isso, este trabalho realiza um estudo sobre o uso da linguagem GraphQL como solução no desenvolvimento de uma ferramenta para que clientes possam buscar e compor estruturas de dados JSON sem a necessidade de entender e/ou escrever código específico à um serviço e interface de aplicação. Permitindo que organizações possam diminuir o tempo e tamanho de resposta sem causar impacto no desenvolvimento de clientes.
}
\palavrasChave{
  GraphQL,
  REST,
  JSON Schema.
}

\paginaresumo
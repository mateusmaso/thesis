\begin{resumo}
  A fim de manter a eficiência da comunicação cliente-servidor sem a necessidade versionamento de APIs Web, serviços têm encontrado dificuldades em realizar mudanças em suas especificações devido ao acoplamento causado pela implementação do código de busca de clientes. No intuito de desenvolver uma solução para o problema de acoplamento, este trabalho realiza um estudo sobre o uso da linguagem GraphQL e formatos de descrição de API para propor um modelo de comunicação cliente-servidor através da automação na execução de consultas de dados. Como resultado, é desenvolvida uma ferramenta prevista pelo modelo para validar sua aplicabilidade e direcionar desenvolvedores de clientes à implementação de código de busca independente de especificação de API. \\ \\
  \textbf{Palavras-chaves}: GraphQL. REST. JSON Hyper-Schema.
\end{resumo}

\textoResumo{  
  Para acomodar a rápida transformação na demanda de dados por clientes de aplicações distribuídas, recentemente tem se discutido novas e eficientes maneiras de se disponibilizar estruturas de dados para consumo de clientes diversificados. Contudo, as atuais formas de comunicação cliente-servidor têm causado dependência na implementação do fluxo de dados e no estilo de arquitetura oferecida por APIs. Isso porque, para o acesso de estruturas de dados, clientes precisam se preocupar em escrever código voltado à especificação atual da interface de acesso, levando a um acoplamento muitas vezes indesejado. Para evitar isso, este trabalho realiza um estudo sobre o uso da linguagem GraphQL como solução no desenvolvimento de uma ferramenta para que clientes possam buscar e compor estruturas de dados JSON sem a necessidade de entender e escrever código específico à API de um serviço ou mais serviços. Essa estratégia visa permitir a redução do tempo e do tamanho de resposta sem causar impacto no desenvolvimento de clientes.
}
\palavrasChave{
  GraphQL,
  REST,
  JSON Schema.
}

\paginaresumo
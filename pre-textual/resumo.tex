\begin{resumo}
  Para acomodar a rápida transformação na demanda por dados em aplicações distribuídas, recentemente tem-se discutido novas maneiras de disponibilizar dados para o consumo eficiente de clientes diversificados. Contudo, as atuais formas de comunicação cliente-servidor têm causado dependência e acoplamento entre o código de busca de dados e a especificação utilizada para acesso da API. Isso porque, para garantir a integridade da comunicação, clientes e serviços precisam estabelecer um contrato de interface de acesso para que não haja mudanças após a implementação do cliente. Para evitar isso, este projeto realiza um estudo sobre o uso da linguagem GraphQL para propor um modelo de comunicação cliente-servidor que não dependa de contrato de API e permita a composição de serviços na busca de dados JSON. Além disso, é desenvolvido uma ferramenta com base no modelo proposto para validar sua aplicabilidade. \\ \\
  \textbf{Palavras-chaves}: GraphQL. REST. JSON Schema.
\end{resumo}
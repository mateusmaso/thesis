\begin{resumo}[Abstract]
  \begin{otherlanguage*}{english}
    In order to keep pace with the increasing changes on data queries executions by client over Web APIs, services have been applying efforts to fine-tune interface methods to maintain efficient client-server communication. However, there are services difficulties in managing changes in the API because, depending on the data fetching code implemented by the clients that access them, there is the possibility of compromising much of the communication. In order to develop solutions for the coupling problem, this work performs a study on the usage of GraphQL language and API description formats to propose a client-server communication model through automation in the execution of data queries. Thus, the model aims to guide client developers to the implementation of data fetching codes independent on API specification. As a result, a tool foresaw by the model is developed to validate its applicability. \\ \\
    \textbf{Key-words}: GraphQL. REST. JSON Hyper-Schema.
  \end{otherlanguage*}
\end{resumo}

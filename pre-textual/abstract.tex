\begin{resumo}[Abstract]
  \begin{otherlanguage*}{english}
  In order to embrace the rapidly transforming data needs from distributed applications, it has been discussed recently efficient ways to expose data for diversified consumption of clients. However, the current client-server communication approach has created dependency and coupling between data fetching code and the API specification used for access. This is because, to ensure communication integrity, clients and services must stablish an interface access agreement so that there won’t be any changes after client's implementation. To solve that, this project studies the usage of GraphQL language to offer a cliente-server communication model that does not depend on API agreements and allows fetching JSON data through service composition. Furthermore, a tool is developed based on the proposed model to validate it's aplicability. \\ \\
    \textbf{Key-words}: GraphQL. REST. JSON Hyper-Schema.
  \end{otherlanguage*}
\end{resumo}

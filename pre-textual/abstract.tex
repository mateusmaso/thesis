\begin{resumo}[Abstract]
  \begin{otherlanguage*}{english}
    In order to keep pace with the increasing transformations by clients on the execution of data queries over Web APIs, services have been making efforts on exposing specifications based on access demand to maintain client-server communication efficiency. However, there are still hitches experienced by services while performing API changes without versioning because, depending on the data fetching code implemented by the clients that access them, there is a chance of compromising a share of this communication. Seeking to develop a solution for the coupling problem, this project conducts a study on the usage of GraphQL language and API description formats to propose a client-server communication model through automation in the execution of data queries. Thus, the model aims to guide client developers to the implementation of data fetching codes independent of API specification. As a result, a tool foresaw by the model is developed to validate its applicability. \\ \\
    \textbf{Key-words}: GraphQL. REST. JSON Hyper-Schema.
  \end{otherlanguage*}
\end{resumo}

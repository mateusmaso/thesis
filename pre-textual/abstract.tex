\begin{resumo}[Abstract]
  \begin{otherlanguage*}{english}
    In order to maintain client-server communication efficiency without the need of versioning Web APIs, services have come across problems while performing changes on their interface access specification due to the coupling caused by clients on the implementation of its fetching code. Seeking to develop a solution for the problem, this project conducts a study on the usage of GraphQL language and API description formats to propose a client-server communication model through automation in the execution of data queries. As a result, a tool foresaw by the proposed model is developed to validate its applicability and guide client-side developers to the implementation of data fetching code independent of API specification. \\ \\
    \textbf{Key-words}: GraphQL. REST. JSON Hyper-Schema.
  \end{otherlanguage*}
\end{resumo}
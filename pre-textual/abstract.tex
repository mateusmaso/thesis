\begin{resumo}[Abstract]
  \begin{otherlanguage*}{english}
  (Arrumar) In order to embrace the growing transformations of flow in which data is access by clients over APIs Web, services that exposes these data had applied efforts in adapt interfaces to  efficient ways to expose data for a diversified consumption of clients. However, the current client implementation approach of the client-server communication model has created dependency and coupling between the client's data fetching code and the API specification used for access. To solve that, this project studies the usage of GraphQL language and API description formats to offer a new cliente-server communication model that avoid coupling due to client fetching code implementation. By automating queries, the model aims to optimize requests, be tolerant of API changes and help fetch data through service composition. Furthermore, a tool is developed based on the proposed model to validate it's aplicability. \\ \\
    \textbf{Key-words}: GraphQL. REST. JSON Hyper-Schema.
  \end{otherlanguage*}
\end{resumo}

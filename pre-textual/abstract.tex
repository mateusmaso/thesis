\textAbstract{
  In order to embrace the rapidly transforming data needs of clients from distributed applications, recently, it has been discussed new and efficient ways to expose data structures to consumption of a diversified amount of clients. However, the current client-server communication approach has created dependency on data-flow implementation and architecture style offered by the API. This is because, when fetching data structures, clients have to worry on writting specific API access code, often leading to unwanted coupling. To solve that, this project will study the usage of GraphQL query language as a solution for building a tool that helps clients to fetch and compose JSON data structures without worrying of reasoning and/or writting specific service and application interface code. Allowing organizations to cut response time and size without causing impact on client-side development.
}
\keywords{
  GraphQL,
  REST,
  JSON Schema.
}

\paginaabstract